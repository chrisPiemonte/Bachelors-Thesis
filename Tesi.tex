\documentclass[a4paper,12pt,oneside]{book}
\usepackage{graphicx}
\usepackage{fancyhdr}
\usepackage[font = scriptsize, bf]{caption}
\usepackage[italian]{babel}
\usepackage[utf8x]{inputenc}
\usepackage[parfill]{parskip}
\usepackage{amsmath, amssymb}
\usepackage{moreverb}
%\usepackage{subfig}
\usepackage{algorithm}
\usepackage{algpseudocode}
\usepackage[usenames,dvipsnames]{color}
\usepackage[swapnames]{frontespizio}
\usepackage{url}
\usepackage{setspace}
\usepackage{eqparbox,array}
\usepackage{siunitx}
\usepackage{subfigure} 
\usepackage{wrapfig}

\renewcommand{\algorithmiccomment}[1]{  //\emph{\textcolor{Gray}{#1}}}


% Sistema i margini per lasciare pi� spazio nella zona di rilegatura
\addtolength{\oddsidemargin}{+1,0cm} 
\addtolength{\evensidemargin}{+1,0cm} 
\onehalfspacing

% Imposta lo stile della prima pagina del capitolo
\fancypagestyle{plain}
{
    \fancyhead{}
    \fancyfoot[LE,RO]{\thepage}
    \renewcommand{\headrulewidth}{0pt}
}

\DeclareMathOperator*{\argmax}{arg\,max}
\newcommand{\compInterfacciaDB}{Data Interface}
\newcommand{\compLoader}{Loader}
\newcommand{\compMatrix}{Matrix Creator}
\newcommand{\compTermsSel}{Terms Selector}
\newcommand{\compPosition}{Position Calculator}
\newcommand{\compClustering}{Clustering Component}
\newcommand{\compEvolution}{Evolution Discoverer}

\hyphenation{ti-me-win-dow}

\graphicspath{{./Immagini/}}


\begin{document}
% Imposta lo stile di intestazione e pi� di pagina della parte iniziale
	% frontespizio
	\begin{frontespizio}
		\Universita{Bari - ``Aldo Moro''}
		\Logo[3.5cm]{logo_uni}
		\Divisione{Facolt� di Informatica}
		\Corso{\\Informatica e Tecnologie per la Produzione del Software}
		\Annoaccademico{2014-2015}
		\Titoletto{Tesi di laurea\\in\\Programmazione II}
		\Titolo{Url2Vec:\\Clustering di pagine in un grafo Web}
		\Candidato[587662]{Christopher Piemonte}
		\NCandidato{Laureando}
		\Relatore{Chiar.mo Prof. Michelangelo Ceci}
		\Relatore{Chiar.mo Prof. Donato Malerba}
		\Correlatore{Dott.ssa Pasqua Fabiana Lanotte}
		\Margini{3cm}{2cm}{2cm}{2cm}
	\end{frontespizio}
	
	\pagestyle{fancy}
	\fancyfoot{}
	\fancyfoot[LE,RO]{\thepage}
	\fancyhead{}
	\renewcommand{\headrulewidth}{0pt}
	\headheight = 15pt
	\frontmatter
	
	% dedica
	\null\vspace{\stretch{1}}
		\begin{flushright}
			\emph{Ai miei genitori ...}
		\end{flushright}
	\vspace{\stretch{2}}\null
	
	% indice
	\tableofcontents
	\listoftables
	\listoffigures
	\newpage
	\color{white}
	a
	\color{black}
%******************************************************************
% Materiale iniziale
%******************************************************************
%\input{MaterialeInizialeFinale/Frontespizio}
%\input{MaterialeInizialeFinale/Colophon}
%\input{MaterialeInizialeFinale/Dedica}
%\input{MaterialeInizialeFinale/Indici}
%\input{MaterialeInizialeFinale/Sommario+Abstract}
%\input{MaterialeInizialeFinale/Ringraziamenti}
% !TEX encoding = UTF-8
% !TEX TS-program = pdflatex
% !TEX root = ../Tesi.tex
% !TEX spellcheck = it-IT

%*******************************************************
% Introduzione
%*******************************************************
\cleardoublepage
\chapter*{Introduzione}

La principale caratteristica dell'Era dell'Informazione è rappresentata dalla possibilità di generare, memorizzare, trasmettere e processare enormi quantità di dati in modo rapido ed economico.

 La disponibilità di una simile quantità di dati, elaborabili automaticamente, ha consentito un forte incremento del processo di generazione e diffusione di conoscenza, utilizzabile per migliorare processi decisionali. Ad oggi, tuttavia, tali risorse non sono appieno sfruttate in tutti i campi e il loro valore potenziale riserva ancora numerose sorprese.

Questo problema è particolarmente sentito nel contesto Web. Il Web infatti può essere considerato la più grande, eterogenea e dinamica sorgente informativa pubblicamente accessibile. Tali caratteristiche rendono il processo di analisi dei dati e estrazione di nuova conoscenza un task altamente complesso e apre nuove sfide e frontiere per l'Informatica.
%L'analisi di dati pubblicamente accessibili nel Web, l'estrazione di pattern e l'utilizzo degli stessi al fine di generare nuova conoscenza definiscono nuove sfide e frontiere per l'Informatica. 
%  La ricerca di pattern nella generazione e nell'utilizzo di nuovi contenuti web e la conoscenza che portano è un'area ancora giovane nell'informatica e in rapida crescita. 
%Con l'aumentare dei dati disponibili sul web e le potenzialmente infinite pagine generate dinamicamente, il bisogno di preprocessare questa informazione sembra scontrarsi con problemi computazionali. Indicizzare o cercare milioni di documenti non-omogenee sul web è diventata una sfida.

%Ogni giorno migliaia di utenti utilizzano il Web per acquisire informazioni ricercando informazioni su motori di ricerca o navigando siti web. Tuttavia, quelle peculiarità che contraddistinguono il Web come dimensione, eterogeneicità e dinamicità generano notevoli difficoltà per gli utenti nel modo di interagire, ricercare ed utilizzare le informazioni.

Un'importante sfida è rappresentata dal problema di organizzare i contenuti e la struttura dei documenti web.
%A dispetto della diversità delle pagine web nella rete, quelle che risiedono all'interno di una particolare organizzazione, spesso, condividono una certa struttura.
In particolare diversi lavori si sono concentrati sul problema del Web Clustering, ossia il processo di raggruppare pagine web in \textit{cluster} cosìcche ogetti simili  possano essere raggruppati nella stessa classe e oggetti dissimili possano essere raggruppati in classi differenti.
Gli obiettivi di questo processo possono essere molteplici: migliorare l'accessibilità alle informazioni e il ritrovamento delle stesse, comprendere i comportamenti di navigazione degli utenti, comprendere come le informazioni si distribuiscono su più pagine web, etc.

In quest'ottica nasce Url2vec, un sistema per il clustering di pagine web che utilizza il contenuto testuale delle pagine web e la struttura ad hyperlink del sito a cui le pagine da raggruppare appartengono, per estrarre cluster di pagine dello stesso tipo semantico (per esempio pagine web di professori, corsi, prodotti, etc.).

Differentemente dagli algoritmi di clustering di pagine web esistenti in letteratura, Url2Vec non considera un sito web come una collezione di documenti testuali indipendenti tra loro, ma cerca di combinare informazioni relative al contenuto con informazioni strutturali, in modo che due pagine web vengano considerate simili se  caratterizzate da una simile distribuzione di termini e abbiano una una simile correlazione nascosta all'interno dei cammini percorribili nel sito web.
%Il clustering di pagine web è un argomento trattato estensivamente in letteratura come un modo di raggruppare pagine  all'interno di cluster omogenei, anche se gran parte del lavoro svolto si basa su un insieme arbitrario di pagine derivanti da molteplici siti differenti. Relativamente poco è stato il lavoro svolto sul clustering di un specifico sito di una determinata organizzazione.
 
 %Infatti, Url2vec combinando e adattando tecniche di Data Mining e di Natural Language Processing, si propone come valida opzione per il clustering di pagine web estraendo informazioni latenti nella struttura degli hyperlink, denotando una correlazione nascosta nei cammini percorribili nel grafo del web.

Le motivazioni alla base dell'implementazione di Url2vec sono state guidate dal voler sfruttare conoscenza già immagazzinata nella risoluzione di problemi specifici in contesti differenti per il quale erano stati ideati, ricavando un trasferimento della conoscenza.
Infatti Url2Vec combina algoritmi tipici dell'area del Natural Language Processing con quelli della Teoria dei Grafi al fine di utilizzare in modo innovativo algoritmi estenti nel contesto Web. 
%L'obiettivo in particolare di questa tesi è estendere lo stato attuale esistente, implementando componenti per realizzare l'estrazione di pagine web correlate e raggrupparle sulla base delle sequenze attraversabili per vistarle, offrendo un diverso punto di vista considerando maggiormente le relazioni invece che il solo contenuto.

Si definisce di seguito la struttura di questo lavoro di tesi.\\
Nel capitolo 1 ci si occuperà di descrivere lo stato attuale, elencando le metodologie utilizzate, e di analizzare nel dettaglio le diverse problematiche da affrontare durante l'analisi dei dati.
Nel capitolo 2 saranno presentati gli obiettivi principali che la metodologia presentata ed il sistema realizzato hanno seguito, descrivendo nel dettaglio le diverse tecniche utilizzate per la realizzazione delle fasi necessarie all'individuazione dei pattern latenti nella struttura del web.
Nel capitolo 3 si parlerà della frontiera attuale dell'Informatica in tali campi confrontando similitudini e spunti di riflessione.
Nel capitolo 4 si descriverà la sperimentazione effettuata,completa di tabelle, grafici e commenti che evidenziano punti di forza e di debolezza individuati per ciascuna delle tecniche utilizzate per le diverse fasi eseguite dal sistema. Soffermandosi sulle novità introdotte con le metodologie presentate e cercando di confrontarle con quelle consolidate.
Infine nel capitolo 5 si parlerà dei possibili miglioramenti alle tecniche ed alle metodologie proposte.



%\pagestyle{scrheadings} 
%\cleardoublepage
%******************************************************************
% Materiale principale
%******************************************************************


% capitoli
	% frontespizio
	
	\frontmatter	
	\mainmatter

	% Imposta lo stile di intestazione e pi� di pagina dei capitoli
	\fancyfoot{}
	\fancyhead{}
	\fancyhead[LE,RO]{\slshape \leftmark}
	\fancyfoot[LE,RO]{\thepage}
	\renewcommand{\headrulewidth}{1pt}
	\renewcommand{\chaptermark}[1]{%
	\markboth{\thechapter.\ #1}{}}


\chapter{Informazioni latenti nel Web}
\label{cap:capitolo1}
% !TEX encoding = UTF-8
% !TEX TS-program = pdflatex
% !TEX root = ../Tesi.tex
% !TEX spellcheck = it-IT

%************************************************

%************************************************

Il progressivo aumento della dimensione del Web e le informazioni in esso contenute fanno di esso la più grande sorgente informativa pubblicamente accessibile. Il Web infatti contiene qualsiasi tipo di informazione in qualsiasi tipo di formato. 
La sua grande eterogeneità rende l'estrazione e il reperimento delle informazioni un task altamente complesso. 
% il più grande insieme di dati da cui poter estrarre informazione velocemente e liberamente. Il Web è la sorgente informativa più eterogenea tra quelle esistenti, data la sua natura, e contenente dati prevalentemente non strutturati o semi-strutturati. Il problema non è sapere se le informazioni ci sono, ma riuscire a trovarle. 

Sebbene a prima vista il Web possa sembrare un insieme disordinato e non strutturato di informazioni distribuite su molteplici pagine web, in realtà è possibile estrarre molteplici correlazioni nascoste tra informazioni all'interno della stessa pagina web e informazioni tra pagine web connesse tramite hyperlink.

%Anche se a prima vista il Web sembra un insieme disordinato di pagine senza nessuna correlazione logica nella struttura, sia interna che nelle relazioni tra queste, in realtà possono essere trovate numerose correlazioni  nascoste. 
Riuscire ad estrarre tali correlazioni e pattern, analizzando la struttura ad hyperlink di cui il Web si compone, permetterebbe di migliorare diverse applicazioni esistenti.

Il problema rimane l'individuazione del procedimento adatto al compito prefissato. Tecniche e metodologie per l'estrazione di conoscenza da grandi quantità di dati sono già state sviluppate nell'area del Data Mining. Trasferire questo sapere nel Web, tuttavia, può non essere semplice e immediato, date la caratteristiche che lo contraddistinguono. 
Di seguito sarà introdotto il contesto in cui si sviluppa la tesi, le varie aree in cui si colloca e da cui attinge le metodologie ed il bagaglio di conoscenza utile alla sintesi di nuovi algoritmi di estrazione di conoscenza dal Web.

\section{Data Mining nel Web}
Il Data Mining è l'insieme di tecniche e metodologie che hanno per oggetto estrazione di informazione utile, di un sapere o di una conoscenza a partire da grandi quantità di dati.

Il concetto di informazione è strettamente legato al contesto in cui si esegue un task di apprendimento, in altre parole un dato può essere interessante o trascurabile a seconda del tipo di applicazione in cui si vuole operare. La fase di estrazione di informazione dai dati per renderla direttamente utilizzabile può variare enormemente dal dominio applicativo. Le differenze sono tali da dover suddividere tali procedimenti in aree diverse, dipendenti dal tipo di dati da cui parte il processo di estrazione. Principalmente le grandi moli di dati possono variare da grandi collezioni di documenti, database, pagine Web ecc, differenziandosi molto nella struttura e nel contenuto,.
\\\\
Il \textbf{Web Mining} è l’applicazione delle tecniche di Data Mining per la scoperta e l’estrazione di conoscenza o di pattern dal World Wide Web.
\\
Le proprietà che caratterizzano il Web e che lo differenziano da altre sorgenti di dati sono:
\begin{itemize}
\item \textbf{Dimensione}: 

Il Web è il primo mezzo di informazione dove il numero di produttori di informazioni è uguale al numero dei consumatori. La quantità dei dati da processare può essere più grande di svariati ordini di grandezza rispetto a tradizionali task di Data Mining. Diventa necessario l'utilizzo di tecniche scalabili, ossia la capacità di un sistema di ''crescere'' ed essere utilizzabile in funzione della crescita dei dati. Nel Data Mining tradizionale, processare un milione di record può essere considerato sopra la media, mentre nel Web Mining dieci milioni di pagine possono non essere abbastanza. 

\item \textbf{Dinamicità}: 
Ogni secondo sono create, distrutte e modificate migliaia di pagine Web. Questo rende il Web una rete di informazioni dinamica, dove la struttura e il contenuto dell'informazione cambiano frequentemente. Monitorare questi cambiamenti rimane un problema importante per molte applicazioni.

\item \textbf{Eterogeneità}:
L'eterogeneità del Web può dipendere sia dal formato delle pagine che dalla contenuto testuale. Nel primo caso, l'eterogeneità è dovuta al fatto che non esiste uno standard di formato, dividendo le pagine Web in tre principali  categorie: \textit{i)}~pagine non strutturate \textit{ii)}~pagine strutturate \textit{iii)}~pagine semi-strutturate.
\\
Le pagine \textit{non strutturate}, anche chiamate \textit{free-text pages}, sono scritte in linguaggio naturale. Su queste possono essere applicate tecniche con un certo grado di affidabilità, denotate dall'arbitrarietà nella valutazione dei risultati.
\\
Le pagine \textit{strutturate} sono normalmente ottenute da sorgenti di dati strutturati (e.g. database). Le tecninche di estrazione sono applicate usando l'individuazione di regole sintattiche.
\\
Le pagine \textit{semi-strutturate} si posizionano al centro delle precedenti. Possiedono infatti sezioni strutturate insieme a testo libero, mostrando un certo livello di struttura nascosto nel testo. L'estrazione può avvenire cercando pattern nei tag HTML, utilizzando i metadati o identificando solo l'informazione strutturata.
\\
In questo caso l'eterogeneità è dovuta al fatto che i contenuti Web sono creati da milioni di persone aventi differente cultura, abilità e linguaggio. Questo significa che le pagine potrebbero contenere la stessa informazione, ma presentata in maniera completamente diversa.

\item \textbf{Connessione}: Il Web è generalmente rappresentato come una rete di informazioni dove i nodi sono le pagine e gli archi sono gli hyperlink. Gli hyperlink fra le pagine di uno stesso sito e quelli fra le pagine di siti diversi, hanno caratteristiche e funzionalità diverse. All'interno del sito servono ad organizzare i contenuti, mentre fra siti diversi sono usati per trasportare autorità alle pagine di destinazione, che avranno prevalentemente contenuti simili o inerenti a quelli della pagina di partenza. In questo caso significa che come persone ci fidiamo del contenuto di queste pagine.

\item \textbf{Rumore}: 
Differentemente da altri mezzi di informazione, la pubblicazione di contenuti è libera e non richiede approvazione. Questo contribuisce all'aumentare del volume e della diversità dell'informazione, ma anche alla creazione di contenuti ridondanti e poco informativi.

\item \textbf{Società virtuale}: La vastità e la proliferazione del Web lo rendono di fatti un enorme Social Network, dove le persone possono comunicare e influenzarsi reciprocamente. Infatti non riguarda solo i dati, le informazioni e i servizi, ma anche le interazioni fra le persone, le organizzazioni o i sistemi.

\end{itemize}

Sulla base di tali caratteristiche, task di Web Mining possono utilizzare tecniche diverse ed essere finalizzate alla scoperta di informazioni differenti. Si possono distinguere principalmente tre categorie:
\begin{itemize}
\item \textit{Web Usage Mining}, è l’applicazione di tecniche di Data Mining per la scoperta di pattern e informazioni utili attraverso l'analisi di log immagazzinati dai web server e contenenti click stream. 
Obiettivo di questo campo è l'apprendimento dell'identità, origine e comportamenti degli utenti che navigano i siti Web al fine di comprendere i loro bisogni e ad offrire loro servizi migliori attraverso una personalizzazione dell'esperienza web.
\item \textit{Web Structure Mining}, consiste nell'estrazione di relazioni sconosciute o nascoste tra pagine web attraverso l'analisi della struttura ad hyperlink di un sito web (anche chiamato ``grafo web''). Questo task verrà analizzato in dettaglio nella Sezione \ref{subsec:webstructure}.
\item \textit{Web Content Mining}, consiste nell'estrazione ed integrazione di informazione utile e precedentemente sconosciuta dal contenuto delle pagine Web. Ricadono in questo campo due principali tipologie di algoritmi: \textit{i)}~algoritmi capaci di raggruppare e classificare pagine web in funzione del loro contenuto testuale o del topic descritto; \textit{ii)}~algoritmi per estrarre pattern strutturati contenuti nelle pagine Web (per esempio liste di prodotti commerciali, professori, etc.).  
Sebbene questi algoritmi possano sembrare molto simili ai più famosi algoritmi di Data Mining e Text Mining, le pagine Web hanno delle peculiarità che non li  rendono direttamente applicabili. Un'importante branca del Web Content Mining è rappresentato dal Web Information Extraction, il cui obiettivo è quello di estrarre dati strutturati da pagine web e integrarli in tabelle relazionali. In questo contesto il Web Content Mining può essere visto quindi come reperimento e immagazzinamento di informazioni dal contenuto testuale.
\end{itemize}


\section{Web Structure mining}
\label{subsec:webstructure}
Sono sempre di più le organizzazioni che disseminano informazioni in rete. Estrarre questi dati è utile in molti domini applicativi perchè permette di ottenere ed integrare dati da diverse fonti, producendo così servizi migliori, informazioni personalizzate o meta-ricerche.

Tuttavia, differentemente dai documenti testuali tradizionali, il contenuto testuale delle pagine Web è arricchito da hyperlink che dividono l'informazione in molteplici ed interdipendenti pagine Web. Questi hyperlink possono essere usati per identificare le entità provenienti dal mondo reale (e.g. pagine di professori, corsi, prodotti) e le relazioni che intercorrono fra di esse costruendo il grafo del sito ed utilizzando tecniche derivanti dalla \textit{teoria dei grafi}. Inoltre molte pagine Web si presentano come combinazione di testo non strutturato e dati strutturati,tipicamente generati dinamicamente da una sorgente sottostante come un database relazionale. Infatti i documenti Web non sono nè strutturati come un database nè completamente non-strutturati come documenti testuali, le tecniche tradizionali di Data Mining o Text Mining non possono essere applicate direttamente. 

Nel primo caso, le tecniche di Data Mining si basano sul presupposto che i dati usati per apprendere un modello condividono uno schema comune, avente delle tabelle ben definite con attributi, colonne, tuple e vincoli. Le pagine Web non hanno questo presupposto perchè contengono dati eterogenei e gli hyperlink definiscono relazioni il cui significato può variare profondamente. Inoltre le pagine Web sono codificate in HTML che, differentemente da altri linguaggi di markup, è stato progettato solo per la visualizzazione (o \textit{rendering}) dei dati. Per questa ragione, il Web può essere considerato un moderno \textit{legacy system}, in quanto una grande quantità di dati non è facilmente accessibile e direttamente manipolabile. 

Nel secondo caso, le tecniche di Text Mining falliscono nell'apprendere modelli accurati perchè: \textit{i)}~richiedono collezioni di documenti scritti in modo consistente; \textit{ii)}~non sono in grado di gestire informazioni complesse con elementi che possiedono diversi ruoli semantici e che forniscono diverse funzionalità. Infatti differentemente dai documenti testuali, le pagine Web hanno molteplici rappresentazioni che forniscono differenti informazioni, quali la rappresentazione testuale del testo HTML e la rappresentazione visuale renderizzata da un web browser. Algoritmi di Text Mining si concentrano sulla rappresentazione testuale ed ignorano la rappresentazione visuale. Di conseguenza, esiste un forte bisogno nel campo dell'informatica di creare approcci e tecniche che usando informazioni testuali, strutturali e visuali sono capaci di estrarre uno schema da dati strutturati ed allineare i dati di conseguenza.
\\\\
Il Web Structure Mining può essere diviso in due tipi:
\begin{itemize}
\item Estrazione di dati strutturati tra pagine web attraverso l'analisi del sito web. In questo caso un sito web è rappresentato come un grafo $G = (V, E)$ dove $V$ è l'insieme delle pagine web e $E$ è l'insieme degli hyperlink.

\item Estrazione di dati strutturati contenuti in una pagina web, analizzandone la struttura ad albero basata su tag HTML ed XML.
\end{itemize}
Tra i più importanti algoritmi di web structure mining troviamo Page Rank~\cite{pagerank} e HITS~\cite{Kleinberg99}, i quali sfruttano la struttura ad hyperlink del Web per assegnare un rank alle pagine, ovvero per restituirle in ordine di importanza relativamente ad una determinata query. 

\section{Rappresentazioni vettoriali di pagine Web}
Reperire informazione dalle pagine Web, quando queste si presentano in forma non-strutturata o semi-strutturata, necessita di passaggi preliminari per rendere i dati processabili dai tradizionali algoritmi di Machine Learning o Data Mining.

La maggior parte delle soluzioni attuali trasformano il contenuto testuale delle pagine Web in uno spazio vettoriale \cite{Turney10}. Questo è motivato dal fatto che task di apprendimento richiedono input che sono matematicamente e computazionalmente convenienti da elaborare, considerando solo un sottoinsieme significativo dei dati in modo da astrarre ed eliminare informazioni ritenute non pertinenti al risultato finale. 

Modelli basati su spazi vettoriali sono fondamentali per task che coinvolgono il calcolo della similarità in quanto oggetti simili sono caratterizzati da rappresentazioni vettoriali simili. Il modello vettoriale \textit{termini-documenti} è uno dei più utilizzati modelli di rappresentazione di documenti testuali nel contesto del Text Mining. In tale modello il valore dell'elemento $i$-esimo in un vettore documento rappresenta il numero di volte che il termine $i$ compare nel documento stesso.

Questi modelli si basano sull'assunzione di indipendenza tra i termini all'interno di un documento e sull'assunzione di indipendenza tra i documenti stessi. Le pagine Web violano, come tutti i documenti testuali, la prima assunzione, inoltre i collegamenti ipertestuali definendo relazioni di interdipendenza tra le pagine stesse comportano la violazione della seconda. Può accadere che pagine prive di testo, completamente vuote (e.g. con contenuti grafici) o ridotte al solo template base del sito, vengano raggruppate nello stesso cluster. Inoltre pagine relative dello stesso tipo (e.g. pagine di docenti) potrebbero essere state create da persone diverse e presentare una distribuzione dei termini notevolmente differente. 

Di conseguenza tener conto di informazioni relative alla struttura del sito Web può arricchire di informazioni utili, linearmente indipendenti a quelle derivanti dal contenuto testuale.

%Enormi sforzi son stati fatti per combinare informazione strutturata e non strutturata presente all'interno di pagine web, al fine di estrarre, indicizzare e reperire nuova conoscenza. 

\subsection{Apprendimento dalla struttura}
Gli algoritmi di apprendimento lavorano tanto bene quanto meglio i vettori ricavati rappresentino bene i dati di partenza. È necessario quindi estrarre rappresentazioni utili dai dati grezzi. Esistono molti modi per ricavare questa correlazione, come reti neurali, matrici di co-occorrenza, dimensionality reduction.
\\\\
L'alternativa proposta in questa tesi consiste nell'utilizzare tecniche di Word Embedding per apprendere rappresentazioni vettoriali dei vertici all'interno del grafo, utilizzati congiuntamente ai vettori del contenuto delle pagine Web.
Il Word Embedding è il nome di un insieme di tecniche per il language modeling e per il Feature Learning nel campo del Natural Language Processing (NLP)\cite{Bengio03}, utilizzate  in collezioni di documenti, dove ad ogni parola viene associato un vettore anche chiamato \textit{Feature Vector}. 
\\\\
Il Word Embedding può essere visto come una funzione parametrizzata 
\begin{equation}
  W : words \to \mathbb{R^n}
\end{equation}
che associa una parola in un dato linguaggio ad un vettore multidimensionale. Un esempio potrebbe essere:
\begin{equation}
  W\left ( "cat" \right ) = \left ( 0.2, -0.4, 0.7, \ldots \right )
\end{equation}
A parole simili corrispondono un vettori simili. Se si cambia una parola con un sinonimo, la validità della frase in esame non cambia (e.g. molti cantano bene $\to$ tanti cantano bene). Questo permette di generalizzare da una frase ad una classe di frasi simili o di capire se una frase è valida, ovvero se  è formulata correttamente.  
\\\\
Questo non significa solo poter scambiare una parola con un sinonimo, ma anche di cambiare una parola con una altra in una classe simile (eg. il muro è rosso → il muro è blu) \cite{Collobert11}. Questo può essere appreso analizzando il contesto della parola da analizzare. Ad esempio ci saranno molti casi in cui sono state osservate frasi valide di questo tipo, quindi cambiando la parola “rosso” con la parola “blu” porterebbe alla creazione una frase ugualmente valida. 
\begin{figure}[htb]
	\centering
	\includegraphics[width = 140mm]{weanalogies.png}
	\caption{Parole con vettori simili}
	\label{similarwords}
\end{figure}

Da questo potrebbe sembrare necessario osservare esempi relativi ad ogni parola per permetterci di generalizzarla. Comprendi tutte le parole che hai già visto, ma non hai già visto tutte le frasi che riesci a capire. Questo è l’approccio delle reti neurali.

\paragraph{Analogie}Il Word Embedding mostra un altra proprietà interessante anche se molto controversa: le analogie. Le analogie tra parole sembrano essere nascoste nella differenza dei loro rispettivi vettori \cite{Mikolov13}. 
\begin{equation}
  W\left ( "woman" \right ) -  W\left ( "man" \right ) \simeq W\left ( "aunt" \right ) -  W\left ( "uncle" \right )
\end{equation}
Da questo si evince che c’è una correlazione tra delle parole e le rispettive forme del genere opposto in quanto appariranno in contesti simile, differenti solo per alcuni dettagli come pronomi o articoli. Stessa cosa per tra singolare e plurale \cite{Mikolov13}.
\begin{figure}[htb]
	\centering
	\includegraphics[width = 100mm]{weanalogies2.png}
	\caption{Alcuni esempi di analogie}
	\label{analogies}
\end{figure}
Queste proprietà possono essere considerate effetti collaterali. Non si è cercato di far apprendere il modello in modo da avere parole simili vicine fra loro. Questo sembra essere un punto di forza delle reti neurali nell’apprendere features che rappresentano bene i dati in modo automatico. invece di apprendere un rappresentazione dei dati specifica ed usarla per diversi task, è possibile apprendere un metodo per associare diversi tipi di dati in una singola rappresentazione. Queste tecniche sono note come Transfer Learning, metodi per applicare la conoscenza già appresa in contesti simili.
\\\\
Un esempio può essere il Word Embedding di parole linguaggi diversi. Dato che parole simili saranno associate a vettori simili, parole con significato simile in una lingua e nell’altra finiranno vicine tra loro, così come i loro sinonimi. È possibile notare che anche parole di cui non si conosceva la traduzione o che avessero significati simili, sono finite vicine tra loro \cite{Zou13}.
\begin{figure}[htb]
	\centering
	\includegraphics[width = 100mm]{englishchinese.png}
	\caption{Visualizzazione con t-SNE di un Word Embedding bilingua. }
	\label{englishchinese}
\end{figure}


\subsection{Word2vec}
\label{word2vec}
Un algoritmo molto famoso di word embedding è il recente word2vec. L’algoritmo usa i documenti per far apprendere una rete neurale, massimizzando la probabilità condizionata del contesto data una parola, applicando il modello appreso ad ogni parola per ricavare il vettore corrispondente e calcolando il vettore della frase facendo la media dei vettori delle parole, costruisce la matrice di similarità delle frasi ed usa PageRank per classificare le frasi nel grafo.
\begin{equation}
   arg\max_{\theta} \prod_{\left ( w, c \right ) \in D} p\left ( c|w; \theta \right )
\end{equation}
L’obiettivo è di ottimizzare il parametro $ \left (\theta \right )$ massimizzando la probabilità condizionata del contesto $\left ( c \right )$ data la parola $\left ( w \right )$. $D$ è l’insieme di tutte le coppie $\left ( w, c \right )$. 
Per esempio: “ho mangiato un \underline{\hspace{1cm}}  al McDonald ieri sera”, molto probabilmente restituirà “Big Mac”.
\\\\
Applicare il modello di ogni parola per ottenere il suo vettore corrispondente (Figura \ref{2w2v})
\begin{figure}[htb]
	\centering
	\includegraphics[width = 100mm]{2w2v.jpg}
	\caption{Vettori corrispondenti alle parole}
	\label{2w2v}
\end{figure}
\\\\
Calcolare il vettore delle frasi facendo la media del vettore delle loro parole (Figura \ref{3w2v})
\begin{figure}[htb]
	\centering
	\includegraphics[width = 100mm]{3w2v.jpg}
	\caption{Vettori corrispondenti alle frasi}
	\label{3w2v}
\end{figure}
\\\\
Costruire la matrice di similarità delle frasi (Figura \ref{4w2v})
\begin{figure}[htb]
	\centering
	\includegraphics[width = 100mm]{4w2v.jpg}
	\caption{Matrice di similarità}
	\label{4w2v}
\end{figure}
\\\\
Infine usare PageRank par classificare le frasi nel grafo.
 (Figura \ref{5w2v})
\begin{figure}[htb]
	\centering
	\includegraphics[width = 100mm]{5w2v.jpg}
	\caption{Assegna uno "score" utilizzando Pagerank}
	\label{5w2v}
\end{figure}

Word2vec è una rete neurale a due layer, sebbene non sia profonda (Deep Neural Network) come spesso definita, trasforma il testo in modo che altre reti neurali possano comprenderlo. Prende in input un corpus di documenti e genera un insieme di vettori: Feature Vectors per ogni parola del corpus. I vettori restituiti sono rappresentazioni numeriche del contesto della singola parola. 
\\\\
Dati abbastanza dati, utilizzo e contesti, Word2vec può apprendere rappresentazioni delle parole altamente accurate, basate sulle apparizioni della parola nei diversi contesti. Queste rappresentazioni possono essere usate per trovare associazioni fra parole o per raggruppare documenti e classificarli per argomento. La similarità fra i vettori può essere misurata attraverso la coseno similarità, dove nessuna similarità è espressa come un angolo di 90 gradi, mentre una similarità totale è data  da un angolo di 0 gradi tra i vettori. Ad esempio il vettore relativo a “Sweden” è uguale al vettore “Sweden” mentre il vettore “Norway” ha una distanza di similarità di $0.760124$.
\\\\
Word2vec può apprendere rappresentazioni principalmente in due modi, o usando il contesto per predire la parola data (metodo conosciuto come “continous bag of word”, o \textbf{CBOW}), o usando una parola per predire il contesto (\textbf{skip-gram}). Per oggetti simili risulteranno rappresentazioni simili. Questo infatti costituisce uno dei primi passi da compiere per effettuare task di Machine Leaarning o Data Mining come ad esempio il raggruppamento di vettori simili in cluster attraverso una qualche funzione di similarità. La traduzione in vettori è necessaria per rendere le informazioni facilmente processabili.


\section{Clustering}
Più precisamente, il clustering consiste in un insieme di tecniche di analisi multivariata dei dati volte alla selezione e raggruppamento di elementi omogenei in un insieme di dati \cite{tryon}. Le tecniche di clustering si basano su misure relative alla somiglianza tra gli elementi. In molti approcci questa similarità (o dissimilarità) è concepita in termini di distanza in uno spazio multidimensionale. La bontà delle analisi ottenute dagli algoritmi di clustering dipende molto dalla scelta della metrica, e quindi da come è calcolata la distanza. Gli algoritmi di clustering raggruppano gli elementi sulla base della loro distanza reciproca, e quindi l'appartenenza o meno ad un insieme dipende da quanto l'elemento preso in esame è distante dall'insieme stesso dividendo gli elementi in più cluster(soft/fuzzy clustering) o in un solo cluster(hard cluster).
Le tecniche di clustering si possono basare principalmente su due "filosofie":
partizionale e gerarchico.

\subsubsection{Partizionale}
Gli algoritmi di clustering di questa famiglia creano una partizione delle osservazioni minimizzando una certa funzione di costo:
\begin{equation}
  \sum_{j=1}^{k} E\left ( C_j \right )
\end{equation}
dove $ k$ è il numero desiderato di cluster, $ C_j$ è il j-esimo cluster e $ E : C \to \mathbb{R^+}$ è la funzione di costo associata al singolo cluster. L'algoritmo più famoso appartenente a questa famiglia è il k-means, chiamato così da MacQueen nel 1967 \cite{MacQueen67}.

\subsubsection{Gerarchico}
Nel clustering gerarchico, invece, è necessario individuare il cluster da suddividere in due sottogruppi. Per questa ragione sono necessarie funzioni che misurino la compattezza del cluster, la densità o la distanza dei punti assegnati ad un cluster. Le funzioni normalmente utilizzate nel caso divisivo sono:
\paragraph{Single-link proximity}
Calcola la distanza tra i due cluster come la distanza minima tra elementi appartenenti a cluster diversi:
\begin{equation}
  D\left ( C_i, C_j \right ) = \min_{x \in C_i, y \in C_j} d\left ( x, y \right )
\end{equation}

\paragraph{Average-link proximity}
Questa funzione calcola la distanza tra i due cluster come la media delle distanze elementi:
\begin{equation}
  D\left ( C_i, C_j \right ) = \frac{1}{|C_i||C_j|}\sum_{x \in C_i, y \in C_j} d\left ( x, y \right )
\end{equation}

\paragraph{Complete-link proximity}
Questa funzione valuta la distanza massima tra due punti interni ad un cluster. Tale valore è noto anche come 'diametro del cluster': più tale valore è basso, più il cluster è compatto:
\begin{equation}
  D\left ( C_i, C_j \right ) = \max_{x \in C_i, y \in C_j} d\left ( x, y \right )
\end{equation}

\paragraph{Distanza tra centroidi}
Questa funzione valuta la distanza massima tra due punti interni ad un cluster. Tale valore è noto anche come 'diametro del cluster': più tale valore è basso, più il cluster è compatto:
\begin{equation}
  D\left ( C_i, C_j \right ) = d\left ( \hat{c_i}, \hat{c_j} \right )
\end{equation}


Nei casi precedenti, $ d\left ( x, y \right )$ indica una qualsiasi funzione distanza su uno spazio metrico.
\begin{figure}[htb]
	\centering
	\includegraphics[width = 100mm]{linkages.jpg}
	\caption{Differenze fra i vari tipi di funzioni distanza}
	\label{linkages}
\end{figure}

\subsection{Graph Clustering}
Un grafo è una coppia ordinata $ G = (V, E)$ di insiemi, con $V$ insieme dei nodi ed $E$ insieme degli archi, tali che gli elementi di $E$ siano coppie di elementi da $V$ da $ E \subseteq V\times V$ segue in particolare che  $|E|\le |V|^2$.
\\\\
I grafi sono oggetti discreti che permettono di schematizzare una grande varietà di situazioni e di processi e spesso di consentirne delle analisi in termini quantitativi e algoritmici.
\\\\
Nello studio di reti complesse, è possibile trovare gruppi di nodi fortemente connessi, che possono essere raggruppati in comunità (potenzialmente sovrapposte). Questa disomogeneità di connessioni suggerisce che esiste una certa divisione naturale all’interno della rete. Nel caso particolare di strutture non sovrapposte, la ricerca di comunità implica la divisione della rete in gruppi di nodi con fortemente connessi internamente e connessioni sparse fra i gruppi.
Una definizione più generale è basata sul principio che coppie di nodi sono più probabilmente connessi se fanno parte della stessa comunità, e meno probabilmente connessi se non condividono la stessa comunità.
\\\\
Le comunità sono molto comuni all’interno delle reti. Le reti sociali includono gruppi di comunità che condividono la posizione, gli interessi, l’occupazione ecc. Essere in grado di individuare queste sotto-strutture all’interno di una rete può fornire indizi  su come come funziona la rete in considerazione o la topologia che influenza i nodi. Questi indizi possono essere utili per implementare algoritmi sui grafi. Molti metodi di community detection sono stati sviluppati con diversi livelli di successo.

\subsubsection{Minimum-cut method}
In questo metodo, la rete è divisa in un numero predeterminato di parti, generalmente della stessa grandezza, scelte in modo che il numero degli archi tra i gruppi è minimizzato. Questo metodo funziona bene in molte applicazioni per le quali è stato ideato, ma non è la scelta migliore per scovare comunità in reti generali, dato che troverà comunità indistintamente dal fatto che queste ci siano o meno e troverà solo un numero fissato di comunità.

\subsubsection{Hierachical-clustering}
Un altro metodo per scovare sotto-strutture conesse nelle reti viene effettuato tramite algoritmi di clustering gerarchico. Con questo approccio si definisce una misura di similarità fra coppie di nodi. Misure comunemente usate sono la coseno similarità, l’indice di Jaccard e la distanza di Hamming fra le righe della matrice di adiacenza. Poi i gruppi di nodi simili vengono raggruppati in comunità.

\subsubsection{Girvan-newman algorithm}
Un altro algoritmo molto utilizzato è quello di Girvan–Newman. Questo algoritmo identifica all’interno della rete, gli archi che uniscono community diverse e li rimuove, isolandole. L’identificazione di tali archi è effettuata applicando una misura nota della teoria dei grafi: la \textbf{betweenness centrality}. Questa assegna un valore ad ogni arco, che è alto tanto più l’arco è attraversato nel cammino più breve (geodesico) che collega due qualsiasi nodi della rete.
\begin{figure}[htb]
	\centering
	\includegraphics[width = 100mm]{betweenness.png}
	\caption{Betweenneess centrality score}
	\label{betweenness}
\end{figure}

\subsubsection{Modularity maximization}
La modularità è una misura che viene attribuita al grafo.  Questa compara la densità all’interno dei cluster con la densità fra di essi. Indica una certa divisione intrinseca e viene utilizzata per conoscere “quanto“ un grafo è separato.
\\\\
Nonostante i suoi svantaggi uno dei metodi più utilizzati per il community detection è la massimizzazione della modularità. Questo approccio scova strutture connesse tramite la ricerca della miglior divisione di una rete in modo che la modularità risulti massimizzata. 
\\\\
Dato che effettuare un confronto su tutte le possibili combinazioni è solitamente impraticabile, gli algoritmi di questa famiglia si basano su metodi di ottimizzazione approssimati quali algoritmi greedy, cioè che cercano di ottenere una soluzione ottima da un punto di vista globale attraverso la scelta della soluzione considerata migliore ad ogni passo locale. Un famoso approccio di questo tipo è il metodo Louvain, che ottimizza le community locali iterativamente, fin quando la modularità globale non può più essere migliorata. 
\\\\
L’accuratezza di questi algoritmi, comunque, è dibattuta, in quanto è stato dimostrato che molte volte fallisce nell’individuare cluster più piccola di una certa soglia, dipendente dalla grandezza della rete. 
\begin{figure}[htb]
	\centering
	\includegraphics[width = 100mm]{modularity.png}
	\caption{Modularity score}
	\label{modularity}
\end{figure}
\subsubsection{Clique-based methods}
Le cricche (cliques) sono sottografi dove ogni nodo è collegato con ogni altro nodo nella cricca. Dato che i nodi non possono essere più connessi di così, non è sorprendente che ci siano molti metodi in community detection che si basano su questo approccio. È da notare che dato che un nodo può far parte di più di una cricca, quindi può far parte di più community contemporaneamente, questi metodi restituiscono strutture sovrapposte. 
Un approccio consiste nel trovare cricche tali che non siano sottografi di altre cricche. Un classico algoritmo per scovare tali strutture è quello di Bron-Kerbosch.



\subsection{Vector Clustering}
Gli algoritmi di clustering in uno spazio vettoriale seguono un altro approccio. Qui viene preso in considerazione la vicinanza (o la distanza) degli elementi rappresentati come punti su un iperpiano. Il feature vector associato può avere grandi dimensioni e può essere ottenuto utilizzando diverse metodologie, dipendentemente dal tipo di dato eleborato. 

\subsubsection{Hierarchical Clustering}
Anche qui il clustering gerarchico è molto utilizzato, seguendo un approccio divisivo (top-down) o agglomerativo (bottom-up), l’idea è quella di unire (o separare) elementi in base alla loro vicinanza, seguendo uno degli approcci già descritti, costruendo così una struttura chiamata dendrogramma che raggruppa gli elementi ad ogni livello. La differenza risiede nel come viene calcolata la distanza.

\subsubsection{Centroid-based clustering}
Nell’approccio basato sui centroidi, i cluster sonorappresentati da un vettore centrale, che non è neccessariamente un membro del dataset. Quando il numero dei cluster è prefissato ad un numero k, la seguente definizione formale può essere applicata: vengono definiti k centroidi e si prosegue assegnando ogni elemento al centroide più vicino, tale che il quadrato delle distanze dal cluster è minimizzato. Molti algoritmi di questa famiglia richiedono che il parametro k sia stabilito in precedenza, che è il loro più grande svantaggio. Inoltre solitamente vengono trovati cluster di grandezza simile, dato che verrà assegnato un elemento al centroide più vicino. Questi metodi partizionano lo spazio dei dati in una struttura conosciuta come diagramma di Voronoi.
\begin{figure}[htb]
	\centering
	\includegraphics[width = 130mm]{voronoi.png}
	\caption{Diagramma di Voronoi}
	\label{voronoi}
\end{figure}
Nonostante questo, rimangono tra degli approcci più utilizzati ed efficaci. Da notare anche la somiglianza concettuale con l’algoritmo di classificazione KNN (k nearest neighbor).

\subsubsection{Distribution-based clustering}
Il raggruppamento avviene analizzando l'intorno di ogni punto dello spazio. In particolare, viene considerata la densità di punti in un intorno di raggio fissato. Si basano sul considerare collegati due punti che si trovano all’interno di una certa distanza limite.
I cluster sono definiti come aree con più alta densità rispetto al resto del dataset. Elementi in un area meno denso sono spesso considerati rumore, quindi come non facenti parte di nessun cluster.
\\\\
Uno degli svantaggi di questi algoritmi è che si aspettano un certo tipo di densità comune a tutti i cluster. Inoltre non eccellono nel scovare cluster presenti in molti dati del mondo reale.

\subsubsection{Document Clustering}
Agisce sempre nello spazio vettoriale, si differenzia principalmente nelle operazioni di pre-processing finalizzate ad ottenere un feature vector utilizzabile. 
Un approccio efficace consiste nel rappresentare i documenti come vettori dove ogni dimensione rappresenta la frequenza di occorrenza di una parola del vocabolario, un insieme di parole precedentemente costruito utilizzando tutte le parole del corpus. 
\begin{figure}[htb]
	\centering
	\includegraphics[width = 100mm]{tdm.png}
	\caption{Matrice termini-documenti. Ogni riga rappresenta un singolo termine ed ogni colonna rappresenta un singolo documento}
	\label{tdm}
\end{figure}

\paragraph{Term-Frequency (TF)}misura quante volte un termine appare in un documento. Dato che ogni documento ha lunghezza differente, è possibile che un termina possa apparire molte più volte nei documenti più lunghi che in quelli più corti. Quindi può essere necessario  dividere la frequenza dei termini per documenti aventi la stessa lunghezza.

\paragraph{Inverse-Document-Frequency (IDF)}un altro aspetto da considerare è a frequenza di un termine in un documento relativamente alla sua presenza globale in tutto il corpus. Tenendo in considerazione solo la frequenza di occorrenza tutti i termini sono considerati ugulamente importanti. Termini che appaiono molte volte in un documento ma meno volte in tutto il corpus potrebbero essere molto più significativi per quel specifico documento e portare molta più informazione, quindi tendono ad essere più importanti. Così come i termini che appaiono molte volte in tutti i dcoumenti del corpus sono spesso poco rilevanti e considerati inutili (stopwords).
\begin{equation}
	idf_i = \log \frac{|D|}{|\left \{ d : t_i \in d \right \}|}
\end{equation}
dove $ |D| $ è il numero di documenti nella collezione, mentre il denominatore è il numero di documenti che contengono il termine $t_i$.
Tale funzione aumenta proporzionalmente al numero di volte che il termine è contenuto nel documento, ma cresce in maniera inversamente proporzionale con la frequenza del termine nella collezione. L'idea alla base di questo comportamento è di dare più importanza ai termini che compaiono nel documento, ma che in generale sono poco frequenti.
\\\\
Altre tecniche utilizzate nel clustering sui documenti sono
\paragraph{LSI} il \textbf{Latent Semantic Indexing} è un metodo di indicizzazione e reperimento che usa una tecnica matematica chiamata decomposizione a valori singolari (SVD) per identificare pattern nelle relazioni tra i termini e i concetti contenuti in una colezione non strutturata di testo. La LSI è basata sul principio che parole che sono usate nello stesso contesto tendono ad avere significato simile. Chiamata così per la sua abilità di correlare semanticamente termini correlati che sono nascosti (latenti) in grandi collezioni testuali. La SVD può venire troncata per task di dimensionality reduction, in modo da diminuire la dimensione del vettore mantenendo comunque il significato.

\paragraph{Coseno similarità} una euristica per la misurazione della similitudine tra due vettori effettuata calcolando il coseno tra di loro. Dati due vettori di attributi numerici, $A$ e $B$, il livello di similarità tra di loro è espresso utilizzando la formula
\begin{equation}
	similarity = \cos \theta = \frac{A\cdot B}{||A||||B||}
\end{equation}
In base alla definizione del coseno, dati due vettori si otterrà sempre un valore di similitudine compreso tra $-1$ e $+1$, dove $-1$ indica una corrispondenza esatta ma opposta (ossia un vettore contiene l'opposto dei valori presenti nell'altro) e $+1$ indica due vettori uguali.
Nel caso dell'analisi dei testi, poiché le frequenze dei termini sono sempre valori positivi, si otterranno valori che vanno da 0 a $+1$, dove $+1$ indica che le parole contenute nei due testi sono le stesse (ma non necessariamente nello stesso ordine) e $0$ che non c'è nessuna parola che appare in entrambi.

\subsection{Algoritmi utilizzati}
Vengono riportati di seguito gli algoritmi di clustering testati sul dataset generato. L'obiettivo della tesi comunque non è verificare la validità di questi ma verificare se la soluzione proposta rappresenti un miglioramento ed una possibile alternativa alle soluzioni più diffuse e consolidate nell'ambito del clustering di pagine Web.
\begin{itemize}
\item \textbf{WalkTrap}
È un approccio basato su Random Walk. L'idea generale è che se vengono generati dei Random Walk sul grafo, i percorsi rimarrano probabilmente all'interno della stessa comunità perchè ci sono meno archi che congiungono comunità diverse. 
\\
L'algoritmo esegue piccoli Random Walk (dipendente da un parametro in input) è usa i risultati per fondere comunità diverse in maniera bottom-up. Tagliando il dendrogramma risultante ad una certa altezza è possibile ricevere il numero di cluster desiderati.

\item \textbf{Fastgreedy}
È un approccio gerarchico bottom-up. Cerca di ottimizzare una funzione di modularità in maniera greedy, euristica attuata effettuando la scelta migliore con le informazioni in possesso ad ogni iterazione.
Inizialmente ogni nodo è una comunità separata e ricorsivamente si procede ad unire i nodi in modo che la fusione porti al massimo aumento di modularità rispetto al valore corrente. 
\\
L'algoritmo finisce quando non è più possibile aumentare la modularità. È un metodo veloce, solitamente usato come primo approccio perchè non ha parametri da modificare.
\item \textbf{K-Means}
Divide il dataset in un numero prefissato $k$ di cluster. Inizialmente vengono scelti casualmente $k$ punti, non necessariamente facenti parte del dataset, chiamati centroidi. Si procede assegnando ogni data point al centroide più vicino e ricalcolando il centroide sulla media aritmetica dei punti contiene. 
\\
Questo processo di assegnazione dei data point e ricalcolo dei centroidi continua fino a quando non avvengono più assegnazioni. I $k$ cluster risultanti saranno quelli restituiti. Anche questo algoritmo rappresenta spesso il punto di partenza nell'analisi di un dataset, in quanto molto spesso porta a buoni risultati ma ha come svantaggio il dover sapere a priori il numero di cluster desiderati.

\item \textbf{DBSCAN}
Deriva da \textit{Density-Based Spatial Clustering of Applications with Noise} è un algortimo basato sulla densità, connettendo regioni di punti con densità sufficientemente alta. Fondamentalmente, un punto $q$  è direttamente raggiungibile da un punto $p$ se non viene superata una data distanza $\epsilon$ e se $p$ è circondato da un numero sufficiente di punti, allora $p$ e $q$ possono essere considerati parti di un cluster. 
\\
Si può affermare che $q$ è density-reachable da $p$ se c'è una sequenza $p_1, p_2,  \ldots, p_n$ di punti con $p_1 = p$ e $p_n = q$ dove ogni $p_{i+1}$ è density-reachable direttamente da $p_i$. Da notare che la relazione density-reachable non è simmetrica (dato che $q$ potrebbe situarsi su una periferia del cluster, avendo un numero insufficiente di vicini per considerarlo un elemento genuino del cluster). Di conseguenza la nozione density-connected diventa: due punti $p$ e $q$ sono density-connected se c'è un punto $o$ tale che sia $o$ e $p$ che $o$ e $q$ sono density-reachable.
\\
Un cluster, che è un sotto-insieme dei punti del database, soddisfa due proprietà:

\textit{i)}Tutti i punti all'interno del cluster sono mutualmente density-connected.
\textit{ii)}Se un punto è density-connected a un altro punto del cluster, anch'esso è parte del cluster.

\item \textbf{HDBSCAN}
Deriva da \textit{Hierarchical Density-Based Spatial Clustering of Applications with Noise}~\cite{Campello15}. Applica DBSCAN variando il valore dell $\epsilon$ ed integra i risultati restituendo cluster che stabilizzano meglio tale valore.
\\
Questo permette ad HDBSCAN di trovare cluster con densità diversa, principale svantaggio di DBSCAN.
\end{itemize}


\section{Data Visualization}
\begin{figure}[h!]
	\centering
	\includegraphics[width = 90mm]{datavisualization.png}
	\caption{La visualizzazione dei dati è un passo fondamentale nell'analisi dei dati.}
	\label{datavisualization}
\end{figure}
Un nota sulla visualizzazione dei dati, campo in crescita data la corrispondente crescita su economie basate sull’informazione e sulla crescita dei dati generati (big data) portata avanti anche da campi relativamente nuovi nel campo dell’analisi dei dati, come Business Analytics, Business Intelligence, Data Science etc.

Tale disciplina è indirizzata a comunicare informazioni in modo chiaro e comprensibile, attraverso grafici, tabelle, diagrammi ecc. La visualizzazione può spesso aiutare ad analizzare e ragionare sui dati, rendendo dati complessi molto più accessibili ed usabili.

Immaginare rappresentazioni di dati in spazi multidimensionali non è intuitivo. Un modo efficace per raggiungere tale scopo può essere effettuato tramite \textit{dimensionality reduction}. Queste consistono nel ridurre la dimensione dei vettori un uno spazio a due o tre dimensioni per poterle rappresentare in modo comprensibile all'occhio umano, che possono avvenire attraverso trasformazioni lineari~\cite{PCA} o non lineari~\cite{vandermaaten08}.
\begin{figure}[htb]
	\centering
	\includegraphics[width = 90mm]{dimred.jpg}
	\caption{Riduzione da tre a due dimensioni.}
	\label{datavisualization}
\end{figure}


\chapter{Il Sistema Url2vec}
\label{cap:capitolo2}
% !TEX encoding = UTF-8
% !TEX TS-program = pdflatex
% !TEX root = ../Tesi.tex
% !TEX spellcheck = it-IT

%************************************************

%************************************************
Il sistema Url2vec si propone come valida alternativa alle metodologie di clustering di siti web basate sul contenuto testuale delle pagine~\cite{Rajaraman11}. A dispetto della diversità delle pagine web nella rete, quelle che risiedono all'interno di una particolare organizzazione, spesso, condividono una certa struttura.
Ad esempio, il sito web di un dipartimento di informatica conterrà pagine riguardante il personale, gli studenti, i corsi, la ricerca che saranno catalogate secondo determinati criteri. Saper sfruttare tale struttura può agevolare notevolmente i task di web mining. Esso infatti aggiunge ulteriore informazione, in modo da estrarre conoscenza strutturata e facilmente processabile. In questa tesi sono state realizzate tre macro componenti.
\begin{itemize}
\item Crawling delle pagine web. Questo è regolato da alcuni parametri che rendono il processo flessibile e altamente modificabile.
\item Costruzione del dataset, ovvero la strutturazione delle informazioni ottenute nel processo precedente, secondo alcuni criteri necessari per le successive elaborazioni.
\item Estrazione di informazione e clustering delle pagine web attraverso la combinazione dell'analisi del contenuto testuale e tecniche di word embedding.
\end{itemize}

Nelle sottosezioni che seguono si descrive il sistema esistente e le tecniche e gli algoritmi utilizzati per la realizzazione degli obiettivi preposti.

\section{Web Crawling per l'estrazione dei dati}
\label{crawling}
Le proprietà che caratterizzano le pagine Web rendono complicato il processo di estrazione di informazioni, soprattutto nel caso in cui i contenuti vengono generati dinamicamente o quando le pagine vengono create o eliminate spesso.

Un Web crawler, chiamato anche web spider o web robot, è un componente software. I suoi obiettivi principali sono:
\begin{itemize}
\item raccogliere il più velocemente ed efficientemente possibile pagine utili, insieme alla struttura ad hyperlink che le collega;
\item aggiornare i contenuti o gli indici dei contenuti di siti Web;
\item copiare tutte le pagine che visita per elaborazioni future, per poi indicizzarle cosi che gli utenti possano trovarle più velocemente;
\item validare gli hyperlink e il codice HTML;
\item eseguire il Web scraping.
\end{itemize}

\begin{figure}[htb]
	\centering
	\includegraphics[width = 100mm]{crawlerarch.png}
	\caption{Architettura di un web crawler}
	\label{crawlerarch}
\end{figure}
Esso esplora una pagina alla volta, analizandone la struttura e gli hyperlink contenuti. Questi sono immagazzinati nella ''frontiera'' che inizialmente vuota, conserva tutte le pagine ancora da esplorare. L'ordine di esplorazione e le politiche di filtraggio degli hyperlink possono variare in base al risultato desiderato. 

\subsection{Proprietà del web crawler}
Nel processo di crawling, data la natura non strutturata del web, è necessaria l'applicazione di numerose tecniche e metodologie per un corretto funzionamento.

\subsubsection{Normalizzazione degli URL}
Il termine di normalizzazione, chiamato anche canonicalizzazione di URL, si riferisce al processo di modifica e standardizzazione di un URL in una maniera consistente, ad esempio alcuni siti Web mettono a disposizione gli stessi file o i medesimi contenuti attraverso URL differenti. 
\\\\
\texttt{http://domain.com/products/page.php?product=smartphone}
\\
\texttt{http://domain.com/products/smartphone.php}
\\\\
\texttt{http://www.domain.edu/courses}
\\
\texttt{http://www.domain.edu/courses/index.html}
\\\\
Le due coppie di URL nell’esempio puntano agli stessi contenuti. Altri casi sono URL che differscono solo per il protocollo (''\texttt{http://}'' o ''\texttt{https://}'') o l'omissione della stringa ''\texttt{www}''. Effettuando la normalizzazione, si sceglie un URL come formato di riferimento per accedere ad un determinato contenuto. Ci sono diversi tipi di normalizzazione che possono essere usati, come la conversione in minuscolo, rimuovere i ''.'' e ''..'' portando gli URL relativi ad URL assoluti, aggiungere slash finali al componente di percorso non vuoto.
\\\\
Una soluzione consiste nella creazione di un dizionario che ha come chiavi gli hashcode del contenuto testuale delle pagine e come valori una lista di tutti i diversi URL che hanno quel contenuto. In questo modo si riduce il problema ad una sola operazione così da annullare le relazioni e i vari di pattern da scovare nall'analisi degli URL per capire se sono la stessa pagina o meno. 

\subsubsection{Ricerca all'interno dello stesso dominio}
Per estrarre informazione e per il successivo processo di clustering delle pagine web, è necessario che le pagine estratte si riferiscano allo stesso dominio, o in altri termini, che appartengano allo stesso sito web.
\\
Questo viene effettuato per sfruttare la struttura gerarchica del dito web rivolto principalmente riuscire a  raccogliere le informazioni nascoste nel grafo e sopratutto negli hyperlink.

\subsubsection{Dimensione della ricerca}
Anche limitando la ricerca ad un solo dominio, la dimensione delle pagine da esplorare, e quindi la grandezza della frontiera, può aumentare considerevolmente o addirittura portare il processo di crawling a divergere.

\begin{figure}[htb]
	\centering
	\includegraphics[width = 100mm]{breadthdepth.png}
	\caption{Differenze tra ricerca in ampiezza e ricerca in profondità}
	\label{breadthsearch}
\end{figure}

È stata utilizzata una ricerca in ampiezza con un limite variabile di profondità. Ponendo un limite all'esplorazione del grafo delle pagine si garantisce la la terminazione del processo di crawling e utilizzando una ricerca in ampiezza si dà priorità alle pagine vicine al nodo radice (comunemente l'homepage). Questa scelta è stata dettata anche dal cercare di evitare le cosiddette ''spider traps''. Queste sono dei meccanismi utilizzati, intenzionalmente o involontariamente, dai server oggetto di crawling, che possono portare ad una generazione dinamica e potenzialmente infinita in URL univoci, e quindi considerati come pagine diverse. Questo può essere evitato non aggiungendo alla frontiera URL che contengono il carattere ''?'', ma questo non è sempre efficace.
\\\\
Inoltre è stato osservato che nei siti web entity-oriented, le informazioni ricercate si trovano quasi sempre nei primi livelli di gerarchia \cite{He13}.

\subsubsection{Restrizione dell'esplorazione}
La restrizione può essere effettuata sulle pagine da esplorare o dalla frequenza di richieste che è possibile effettuare.
\\\\
Il robots exclusion protocol è uno standard che consiste in un file (robots.txt), posto alla radice della gerarchia di un sito Web. In pratica, il file indica le regole utilizzate dai crawler per applicare restrizioni di analisi sulle pagine di un sito web. 
\\\\
Molte volte i crawler non sono ben accetti, in quanto possono rallentare pesantemente la navigazione. Se server effettua dei controlli sulle richieste ricevute e queste hanno una frequenza troppo elevata o seguono uno schema riconducibile ad una macchina queste possono essere ignorate. Può capitare che richieste continue e non curanti delle restrizioni imposte portino a bandire l'indirizzo IP del servizio trasgressore.\\
Per evitare tali conseguenze è opportuno seguire una certa etica nell'operazione di crawling.

\subsection{Estrazione delle liste}
Riuscire a strutturare dati non strutturati può rivelarsi ostico. Molti tentativi, più o meno efficaci, sono stati effettuati a tale proposito.
\\
In questa tesi uno degli obiettivi è stato quello di prendere in considerazione i collegamenti all'interno delle pagine web, seguendo i percorsi generati dalla concatenazione di più hyperlink. Questa scelta è stata effettuata sulla base dei recenti progressi nel campo del natural language processing (NLP)~\cite{Turian10}. Esplorando la struttura del grafo del web, data la forte connessione che esiste fra i suoi nodi, estrarre informazione può rivelarsi un operazione tutt'altro che banale. 
\\\\
È stato introdotto il concetto di \textbf{lista}. 
Per permettere una migliore visualizzazione dell’informazione descritta, quasi tutte le pagine web vengono formattate utilizzando regole CSS. Di conseguenza, per poter effettuare correttamente questo task, bisogna prima elaborare la pagina Web con tutte le informazioni grafiche sui nodi HTML e solo successivamente si può procedere alla loro estrazione. Grazie alle informazioni ricavate dall’HTML della pagina Web e dalla posizione e dimensione dei singoli nodi, è possibile stabilire una struttura gerarchica ad albero dei nodi che la compongono. Questa struttura gerarchica ad albero permette di scoprire i record, ovvero dati strutturati (ad esempio provenienti da database) che sono allineati orizzontalmente, verticalmente e anche strutturalmente (elementi HTML ul, li, . . . ); permette quindi di individuare dei gruppi di record, ovvero delle liste di record.
\\
L’individuazione delle liste di raggiungere pagine che contengono elementi simili a quelli contenuti nella pagina che si sta visualizzando. 
In questo modo le pagine dovrebbero essere accessibili solo attraverso percorsi predeterminati e non in maniera fortemente connessa. I nodi e gli archi risultanti risultano così un sottoinsieme di quelli originali.

\section{Costruzione del dataset}
I dati estratti nel processo vanno organizzati e ampliati in modo da garantire l'accesso e l'elaborazione in maniera agevole.
\\
Il processo di crawling restituisce il grafo delle pagine web e il contenuto testuale di ogni pagina esplorata a queste va aggiunta la generazione delle sequenze. Inoltre per un minor spreco di risorse si è optato per la conversione degli URL in codici, ovvero una associazione univoca fra un URL e un codice (e.g. un numero) molto più corto, così da risparmiare tempo di elaborazione e spazio di archiviazione. Segue una analisi sul processo di generazione delle sequenze.


\subsection{Generazione delle sequenze}
Le sequenze rappresentano un percorso che un attraversatore casuale della rete seguirebbe cliccando su un hyperlink a caso fra tutti quelli disponibili nella pagina corrente (o eventualmente nelle liste). Questi percorsi, chiamati \textbf{random walk} (o passeggiate aleatorie), sono stati largamente utilizzati in molti algoritmi sui grafi e sul web~\cite{aldous14} in quanto buone approssimazioni di comportamenti casuali. Il problema di questa tecnica applicata al web si presenta quando l'attraversatore casuale arriva ad una pagina priva di hyperlink. La soluzione più diffusa consiste nell'effettuare un ''salto'' verso una qualsiasi altra pagina quando non ci sono outlink da seguire. 
\\\\
Qui il problema non si pone, in quanto le sequenze generate hanno una lunghezza fissata prima dell'esecuzione e se la generazione dovesse bloccarsi, la sequenza risultante sarà solo più piccola. Questa scelta è dovuta dal fatto che l'informazione cercata scaturisce da percorsi reali di navigazione e non necessita una lunghezza obbligatoria da rispettare, in quanto le sequenze possono essere viste come frasi di un testo, dove le parole sono gli URL.
\\\\
Per motivi di sperimentazione sono stati implementati tre tipi diversi di random walk, utilizzabili modificando i parametri di esecuzione dell'algoritmo.

\subsubsection{Random Walk}
Questo è il caso standard, ovvero si parte da un nodo casuale del grafo e si segue ogni volta un arco a caso fra quelli disponibili, fino al raggiungimento della lunghezza prefissata o all'impossibilità di proseguire.
\\
Da notare che questo processo e il precedente non sono completamente separati, in quanto la scelta di un nodo casuale e la consecutiva traiettoria, possono portare all'esplorazione di pagine non precedentemente visitate. È quindi necessario mantenere aggiornato il grafo immagazzinato.
\begin{figure}[htb]
	\centering
	\includegraphics[width = 100mm]{randomWalkongraph.png}
	\caption{Random walk sul grafo. }
	\label{englishchinese}
\end{figure}
\subsubsection{Random Walk con partenza fissa}
Qui l'unica differenza consiste nel punto di partenza del cammino. Infatti si può partire in un nodo prefissato del grafo (generalemnte l'homepage di un sto web), in modo da esplorare più percorsi possibili avente quel nodo come origine.

\subsubsection{Random Walk attraverso le Liste}
Qui invece si può seguire uno dei due approcci precedenti, ma con il vincolo delle liste, quindi limitando la camminata ad un sottoinsieme di quella precedente.

\begin{algorithm}[H]
\caption{Crawling BFS}
\begin{algorithmic}

	\State $url\gets homepage$;	\Comment{Pagina da cui iniziare}
	\State $queue\gets empty$ \Comment{Coda per la BFS}
	\State $analyzedVertex.add(url)$;	\Comment{Insieme di url già visitati}
	\State $maxDepth url$;	\Comment{Massima profondità di esplorazione}
 	
 	\vspace*{+0.5cm}
 	
	\State \textbf{Begin}
	\While{$queue \neq empty$} 
		\State $urlToAnalyze\gets queue.dequeue()$
		
		\If{$urlToAnalyze.depth \leq maxDepth$}
			\State $outlinks \gets urlToAnalyze.getOutlinks()$
		\Else
			\State $urlToAnalyze$
		\EndIf
		\If{$outlinks \neq null$}
			\State $outlinks \gets urlToAnalyze.getOutlinks()$
			\State $analyzedVertex.add(outlinks)$
			\State $serialize(urlToAnalyze)$
			\For{\textbf{each} link in outlinks}
				\State $serialize(link)$
				\State $queue.enqueue(link, urlToAnalyze + 1)$
			\EndFor
		\EndIf
		
	\EndWhile
	\State \textbf{End}
\end{algorithmic}
\end{algorithm}

\begin{algorithm}[H]
\caption{Generazione delle sequenze}
\begin{algorithmic}

	\State $numRandomWalks$;	\Comment{Numero di Random Walk da generare}
	\State $lengthRandomWalks$ \Comment{Lunghezza dei Random Walk}
 	
 	\vspace*{+0.5cm}
 	
	\State \textbf{Begin}
	\State $node \gets \Call{randomnode}{}$
	\For{$i \gets 0 \to numRandomWalks$} 
		\State $sequence.add(node)$
		
		\While{$sequence.length \leq lengthRandomWalks$}
			\If{$node.hasOutlinks()$}
				\State $node \gets node.getOutlinks(RandomIndex)$
				\State $sequence.add(node)$
			\Else
				\State $break$
			\EndIf
		\EndWhile
		\State $serialize(sequence)$
		
	\EndFor
	\State \textbf{End}
\end{algorithmic}
\end{algorithm}


\subsection{esempio di dataset}
Di seguito sono elencati file generati nel processo di crawling e di generazione delle sequenze. Gli esempi riportati di seguito sono stati ottenuti analizzando il sito del dipartimento di informatica di Urbana, IL: \texttt{www.cs.illinois.edu}

\subsubsection{urlsMap.txt}
Contiene la associazioni fra gli URL e il relativo codice identificativo. Questo è dovuto dalle ragioni spiegate in precedenza, ovvero ridurre i tempi di elaborazione e spazio di archiviazione. 
\\\\
\texttt{
http://cs.illinois.edu,3\\
http://cs.illinois.edu/prospective-students,4\\
http://cs.illinois.edu/current-students,5\\
http://cs.illinois.edu/courses,6\\
http://cs.illinois.edu/alumni,7\\
http://cs.illinois.edu/research,8\\
http://cs.illinois.edu/news,9\\
http://cs.illinois.edu/partners,10\\
http://cs.illinois.edu/about-us,11\\
. . .\\
}
\subsubsection{vertex.txt}
Contiene il contenuto testuale di ogni pagina esplorata. Ogni riga è quindi formata da il codice identificativo di un URL e il relativo contenuto.
\\\\
\texttt{
1	department of computer science at illinois engineering at ...\\
2	prospective students department of computer science at ...\\
3	current students department of computer science at ...\\
4	courses department of computer science at illinois ...\\
5	alumni department of computer science at illinois ...\\
6	research department of computer science at illinois    ...\\
7	news department of computer science illinois engineering ...\\
8	partners department of computer science at illinois ...\\
9	about us department of computer science at illinois ...\\
. . .\\
}
\subsubsection{edges.txt}
Questo è il file principale per la generazione delle sequenze, qui sono immagazzinate tutte le relazioni fra le pagine, ovvero gli archi che le collegano. 
\\\\
\texttt{
1	1\\
1	2\\
1	3\\
1	4\\
1	5\\
1	6\\
1	7\\
1	8\\
1	9\\
. . .\\
}
\subsubsection{sequencesIDs.txt}
Contiene le sequenze generate. I codici relativi alle pagine web sono separati da '' -1 '' e la linea finisce con un '' -2 ''. Da notare che sono riportate le sequenze che partono da un nodo casuale del grafo.
\\\\
\texttt{
137 -1 2 -1 27 -1 8 -1 52 -1 53 -1 8 -1 8 -1 10 -1 13 -1 -2\\
506 -1 5 -1 14 -1 11 -1 6 -1 2 -1 27 -1 114 -1 111 -1 11 -1 -2\\
424 -1 4 -1 12 -1 6 -1 8 -1 53 -1 4 -1 7 -1 12 -1 8 -1 -2\\
616 -1 5 -1 6 -1 8 -1 8 -1 9 -1 1 -1 21 -1 6 -1 3 -1 -2\\
51 -1 7 -1 7 -1 38 -1 38 -1 25 -1 103 -1 27 -1 113 -1 12 -1 -2\\
429 -1 10 -1 3 -1 6 -1 4 -1 11 -1 8 -1 9 -1 9 -1 3 -1 -2\\
783 -1 421 -1 5 -1 9 -1 7 -1 5 -1 2 -1 8 -1 2 -1 24 -1 -2\\
506 -1 5 -1 8 -1 52 -1 53 -1 25 -1 40 -1 13 -1 11 -1 13 -1 -2\\
638 -1 63 -1 153 -1 62 -1 63 -1 152 -1 63 -1 155 -1 13 -1 7 -1 -2\\
. . .\\
}
\subsubsection{sequencesIDsFromHomepage.txt}
Contiene le sequenze generate che partono da uno stesso nodo. La generazione di questo file avviene esplicitando il nodo di origine di ogni sequenza nella fase di generazione.
\\\\
\texttt{
1 -1 8 -1 2 -1 14 -1 2 -1 2 -1 10 -1 14 -1 66 -1 3 -1 -2\\
1 -1 18 -1 39 -1 8 -1 8 -1 24 -1 1 -1 4 -1 7 -1 25 -1 -2\\
1 -1 23 -1 -2\\
1 -1 16 -1 10 -1 3 -1 29 -1 29 -1 4 -1 25 -1 97 -1 108 -1 -2\\
1 -1 20 -1 20 -1 1 -1 11 -1 3 -1 2 -1 9 -1 10 -1 13 -1 -2\\
1 -1 25 -1 48 -1 48 -1 44 -1 42 -1 4 -1 7 -1 38 -1 38 -1 -2\\
1 -1 25 -1 115 -1 4 -1 11 -1 11 -1 1 -1 2 -1 26 -1 13 -1 -2\\
1 -1 24 -1 4 -1 9 -1 60 -1 56 -1 6 -1 25 -1 113 -1 116 -1 -2\\
1 -1 21 -1 25 -1 135 -1 32 -1 13 -1 4 -1 25 -1 149 -1 59 -1 -2\\
. . .\\
}

\section{Clustering delle pagine web}
In questa sezione si parlerà della soluzione proposta come alternativa alle normali tecniche di clustering delle pagine web basate esclusivamente sul contenuto testuale. Infatti il fulcro del sistema è basato su metodologie nate nel campo del \textbf{Natural Language Processing} ma applicate nel contesto web, in modo da aggiungere ulteriore informazione utile. Qui sarà approfondita l'idea alla base del lavoro svolto e le diverse opzioni che l'algoritmo mette a disposizione. 
L'algoritmo prende in input il grafo di un sito web ele sequenze di random walk generate restituisce rappresentazioni vettoriali per ogni pagina.
La fase di sperimentazione vera e propria sarà approfondita nel capitolo successivo.

\subsection{Algoritmi utilizzati}
Vengono riportati di seguito gli algoritmi di clustering testati sul dataset generato. L'obiettivo della tesi comunque non è verificare la validità di questi ma verificare se la soluzione proposta rappresenti un miglioramento ed un possibile alternativa alle soluzioni più usate e consolidate nell'ambito del clustering di pagine web.
\begin{itemize}
\item \textbf{WalkTrap}
È un approccio basato su Random Walk. L'idea generale è che se vengono generati dei Random Walk sul grafo, i percorsi rimarrano probabilmente all'interno della stessa comunità perchè ci sono meno archi che congiungono comunità diverse. 
\\
L'algoritmo esegue piccoli Random Walk (dipendente da un parametro in input) è usa i risultati per fondere comunità diverse in maniera bottom-up. Tagliando il dendrogramma risultante ad una certa altezza è possibile ricevere il numero di cluster desiderati.

\item \textbf{Fastgreedy}
È un approccio gerarchico bottom-up. Cerca di ottimizzare una funzione di modularità in maniera greedy, euristica attuata effettuando la scelta migliore con le informazioni in possesso ad ogni iterazione.
Inizialmente ogni nodo è una comunità separata e ricorsivamente si procede ad unire i nodi in modo che la fusione porti al massimo aumento di modularità rispetto al valore corrente. 
\\
L'algoritmo finisce quando non è più possibile aumentare la modularità. È un metodo veloce, solitamente usato come primo approccio perchè non ha parametri da modificare.
\item \textbf{K-Means}
Divide il dataset in un numero prefissato $k$ di cluster. Inizialmente vengono scelti casualmente $k$ punti, non necessariamente facenti parte del dataset, chiamati centroidi. Si procede assegnando ogni data point al centroide più vicino e ricalcolando il centroide sulla media aritmetica dei punti contiene. 
\\
Questo processo di assegnazione dei data point e ricalcolo dei centroidi continua fino a quando non avvengono più assegnazioni. I $k$ cluster risultanti saranno quelli restituiti. Anche questo algoritmo rappresenta spesso il punto di partenza nell'analisi di un dataset, in quanto molto spesso porta a buoni risultati ma ha come svantaggio il dover sapere a priori il numero di cluster desiderati.

\item \textbf{DBSCAN}
Deriva da \textit{Density-Based Spatial Clustering of Applications with Noise} è un algortimo basato sulla densità, connettendo regioni di punti con densità sufficientemente alta. Fondamentalmente, un punto $q$  è direttamente raggiungibile da un punto $p$ se non viene superata una data distanza $\epsilon$ e se $p$ è circondato da un numero sufficiente di punti, allora $p$ e $q$ possono essere considerati parti di un cluster. 
\\
Si può affermare che $q$ è density-reachable da $p$ se c'è una sequenza $p_1, p_2,  \ldots, p_n$ di punti con $p_1 = p$ e $p_n = q$ dove ogni $p_{i+1}$ è density-reachable direttamente da $p_i$. Da notare che la relazione density-reachable non è simmetrica (dato che $q$ potrebbe situarsi su una periferia del cluster, avendo un numero insufficiente di vicini per considerarlo un elemento genuino del cluster). Di conseguenza la nozione density-connected diventa: due punti $p$ e $q$ sono density-connected se c'è un punto $o$ tale che sia $o$ e $p$ che $o$ e $q$ sono density-reachable.
\\
Un cluster, che è un sotto-insieme dei punti del database, soddisfa due proprietà:

\textit{i)}Tutti i punti all'interno del cluster sono mutualmente density-connected.
\textit{ii)}Se un punto è density-connected a un altro punto del cluster, anch'esso è parte del cluster.

\item \textbf{HDBSCAN}
Deriva da \textit{Hierarchical Density-Based Spatial Clustering of Applications with Noise}~\cite{Campello15}. Applica DBSCAN variando il valore dell $\epsilon$ ed integra i risultati restituendo cluster che stabilizzano meglio tale valore.
\\
Questo permette ad HDBSCAN di trovare cluster con densità diversa, principale svantaggio di DBSCAN.
\end{itemize}

\subsection{NLP nel web: URL embedding}
I recenti algoritmi di word embedding, riescono a tenere in considerazione sempre più fattori e quindi restituire vettori sempre più accurati. In particolare vengono incluse informazioni riguardanti il contesto di una parola, ovvero le altre parole che sono contenute all'interno delle frasi in cui compare quella sotto esame. Questa informazione riesce ad estrarre correlazioni nascoste e raggruppare i termini in classi. Può sembrare banale ma diversamente dal text clustering tradizionale, dove le parole vengono considerate al livello di documento, qui vengono considerate a livello di frase. Un esempio può riguardare il controverso caso delle analogie. 
\\\\
È stato osservato che correlazioni nascoste possono trovarsi nella differenza tra coppie di vettori~\cite{Mikolov13}, come nel caso di parole simili ma con leggere differenze come il genere o il numero, infatti esse appaiono in frasi tendenzialmente identiche, ma con delle piccole differenze. In questo caso l'informazione può essere interpretata come il genere o il numero.
\\\\
Informazioni simili possono essere trovate anche nel campo del web, analizzando i percorsi di URL come se fossero frasi. È ormai consolidata la questione dell'autorità di alcune pagine~\cite{Kleinberg99}
e di come queste abbiano un maggior numero di link in entrata. Alcune pagine tuttavia saranno accessibili prevalentemente attraverso alcuni percorsi e si troveranno quindi in un contesto con elementi simili. Il problema adesso è riuscire a dare un senso a queste correlazioni e dargli un significato utilizzabile. Nello sviluppo di Url2vec tuttavia sono emerse corrispondenze abbastanza naturali, come le pagine dei professori con i relativi corsi insegnati o i laboratori di afferenza. Queste relazioni sono dovute al fatto che il grafo di un sito web è fortemente connesso, ma la maggior parte delle volte in cui appare un determinato corso di studio appare anche il relativo docente e viceversa. 

\subsection{Differenza tra parole ed URL}
La sola analisi delle sequenze comunque può risultare limitata. Gli URL non sono parole, in quanto ad essi è associata ulteriore informazione, ovvero il contenuto testuale. Questa proprietà si è rilevata molto utile, infatti combinando le informazioni ricavate esaminando tutti e due gli aspetti, il grado di accuratezza del processo di clustering è salito in modo significativo. 



\chapter{Related works}
\label{cap:capitolo3}
% !TEX encoding = UTF-8
% !TEX TS-program = pdflatex
% !TEX root = ../Tesi.tex
% !TEX spellcheck = it-IT

%************************************************

%************************************************

Il clustering di pagine web è un argomento trattato estensivamente in letteratura che ha come obiettivo raggruppare pagine all'interno di cluster omogenei, anche se gran parte del lavoro svolto si basa su un insieme arbitrario di pagine derivanti da molteplici siti differenti. Relativamente poco è stato il lavoro svolto sul clustering di un specifico sito di una determinata organizzazione.
\\\\
Principalmente si dividono sulla base dei criteri che utilizzano per raggruppare le pagine, ovvero se considerano la semantica, la struttura o l'utilizzo.

\subsubsection{Semantica delle pagine web}
Come proposto in \cite{Cooley03}, questo approccio suggerisce l'utilizzo di informazioni semantiche per migliorare il processo di estrazione, necessitano quindi meta-informazioni aggiuntive sulla struttura e sulla gerarchia. Le foglie nel livello più basso sono le pagine web, che sono poi raggruppate in cluster sulla base di una qualche affinità semantica. 
\\
Gerarchie semantiche possono essere definite seguendo diversi criteri, che dipendono dagli obiettivi e dalle analisi. Le informazioni aggiuntive necessarie possono essere basate sul contenuto delle pagine web, come ad esempio i metadati, o sapere se è stato un particolare tool o processo nella creazione delle pagine.

\subsubsection{Struttura interna delle pagine web}
Questi prendono in considerazione la struttura interna di una pagina analizzandone il DOM. La struttura ad albero di una pagine HTML è stata usata per la segmentazione di una pagina ed utilizzare i collegamenti tra queste per raggrupparle \cite{Lin10} o per scovare pattern \cite{Kudelka08} ricorrenti, utili nel misurare la similarità.

\subsubsection{Struttura del grafo del web}
Utilizza il grafo del web, dove le pagine sono i nodi e gli archi sono i collegamenti fra queste.
\\
In questo caso il clustering delle pagine web diventa il partizionamento del grafo. Soluzioni proposte \cite{Luxburg07} si basano sul dividere tale struttura in sotto-grafi, tramite una funzione che minimizza il numero di archi tra cluster e massimizza il numero degli archi tra i nodi di un cluster.

\subsubsection{Web usage clustering}
Possono essere estratti pattern di utilizzo dai web log ed essere utilizzati per predire il comportamento futuro degli utenti, per raggruppare le pagine in cluster in base agli interessi in comune o per pesare gli archi del grafo, in modo da combinare diversi approcci \cite{Shahabi97}. Lavori recenti si stanno dirigendo sempre più sullo web usage mining, quindi raggruppando pagine web prevalentemente in base all'utilizzo da parte degli utenti, in modo da personalizzare le risposte alle query immesse nei motori di ricerca~\cite{Crabtree06}.


\section{In-domain clustering}
Molto del lavoro svolto sull'argomento si focalizza sul clustering di pagine web provenienti da siti diversi basandosi su di un approccio specifico piuttosto che un altro. Un caso interessante è il sistema SiteMap Generator (SMG)~\cite{Lin11} che mette in pratica un approccio ibrido, analizzando sia la struttura interna che quella esterna. Infatti esso divide la pagina in blocchi, li classifica sulla base della loro rilevanza ed infine analizza i collegamenti che ci sono fra blocchi, eventualmente anche di pagine diverse. In seguito i blocchi con alta frequenza di occorrenza e un alto valore hub, ottenuto nell'ultima fase tramite l'algoritmo HITS, vengono utilizzati per generare la sitemap.
\\
In SMG l'obiettivo è dunque la costruzione della sitemap, estraendo dalla struttura delle pagine di uno stesso dominio gli hyperlink necessari.

\section{Random walk come frasi}
Nell'ambito della social networks analysis, DeepWalk \cite{Perozzi14} propone una metodologia interessante attraverso l'analisi dei grafi. Dato un grafo, vengono generati random walk di piccola lunghezza. Questi vengono poi trattati come frasi, e applicando tecniche di Natural Language Processing viene stimata la verosimiglianza che specifiche sequenze di parole (in questo caso i nodi del grafo) appaiano nel corpus, ovvero l'insieme dei random walk generati. Queste vengono poi mappate in uno spazio vettoriale.
\\
Questo approccio viene applicato nell'ambito dei social network per apprendere rappresentazioni sociali dei vertici. La parte interessante è l'applicazione di tecniche consolidate per la risoluzione di problemi differenti.

\section{Sviluppi recenti}
Articoli recenti si allontanano dalla classica rappresentazione vettoriale dei documenti, basata sula frequenza di occorrenza delle parole all'interno dei documenti. Questa può evidenziare una similarità tra i documenti non esaustiva. Può rivelarsi utile disambiguare le query immesse nei motori di ricerca o assegnare una misura di rilevanza di un documento ad un dato argomento.



\chapter{Sperimentazione}
\label{cap:capitolo4}
% !TEX encoding = UTF-8
% !TEX TS-program = pdflatex
% !TEX root = ../Tesi.tex
% !TEX spellcheck = it-IT

%************************************************

%************************************************
In questo capitolo si descriverà l'esecuzione della sperimentazione. QUesta si è svolta confrontando i risultati ottenuti attraverso l'applicazione di diversi algoritmi di clustering su dataset ottenuti da differenti rappresentazioni. L'algoritmo è stato testato effettuando il crawling di $3$ siti web, che sono stati poi etichettati manualmente per ricavare le metriche necessarie alla valutazione della tesi proposta.
\begin{itemize}
\item Il sito web del dipartimento di Computer Science dell'università di Urbana, IL: \texttt{cs.illinois.edu}
\item Il sito web del dipartimento di Computer Science dell'università di Stanford, CA: \texttt{cs.stanford.edu}
\item Il sito web del dipartimento di Computer Science del  Massachusetts Institute of Technology a Cambridge, MA: \texttt{eecs.mit.edu}
\end{itemize}
Si partirà pertanto dai dati su cui quest'ultima è stata effettuata, proseguendo con la scelta delle modalità di esecuzione più interessanti e concludendo con una serie di tabelle e grafici contenenti i risultati ottenuti. 

\section{Decrizione del dataset}
I dataset utilizzati per la sperimentazione sono stati creati eseguendo il processo di crawling e generazione delle sequenze sui seguenti siti:


\paragraph{\texttt{http://cs.illinois.edu/:}}
Il crawling è stato lanciato con profondità massima $10$, e immagazzinando un massimo di $10.000$ termini per ogni pagina esplorata. Dal grafo sono state poi generate $10.000$ sequenze di lunghezza massima $10$. Sono state collezionate $728$ pagine web.


\paragraph{\texttt{http://cs.stanford.edu/:}}
 Il crawling è stato effettuato con profondità massima $10$, e immagazzinando un massimo di $10.000$ termini per ogni pagina esplorata. Dal grafo sono state poi generate $10.000$ sequenze di lunghezza massima $10$. Il processo ha collezionato $1458$ pagine web.

\paragraph{\texttt{http://eecs.mit.edu/:}}
Il crawling è stato lanciato con profondità massima $10$, e immagazzinando un massimo di $10.000$ termini per ogni pagina esplorata. Dal grafo sono state poi generate $10.000$ sequenze di lunghezza massima $10$. Sono state collezionate $1745$ pagine web.

\section{Configurazioni}
Dalla scelta di apprendere le rappresentazioni vettoriali delle relazioni invece di utilizzare algoritmi di Community Detection del grafo, sono derivati dei vantaggi. Innanzitutto gli algoritmi di partizionamento dei grafi hanno complessità NP-completa, ovvero necessitano di un tempo superpolinomiale nella dimensione dell'input. Nel contesto del Web Mining la dimensione del dataset può crescere enormemente ed avere soluzioni più efficienti costituisce senz'altro una priorità. 
\\
In tutti i casi di seguito riportati sono stati generati grafi sia in modalità classica, che attraverso l'estrazione delle liste. Nel primo caso i dataset saranno chiamati \textbf{''nc''} (no-costraint, senza vincoli), mentre nel secondo caso \textbf{''lc''} (list-costraint, con il vincolo delle liste. Dove omessa, la configurazione è rimasta invariata.

\subsection{Community Detection}
Sono state applicate metodologie derivanti dalla teoria dei grafi per l'estrazione di strutture connesse all'interno del grafo web. Questo può essere utile nella individuazione di community all'interno di grafi come ad esempio social network. La divisione del grafo in community può essere effettuata seguendo diversi approcci, inoltre la rete costruita su di un tipico sito web è caratterizzata solitamente da un numero elevato di collegamenti tra pagine che suggeriscono l'utilizzo di alcune tipologie piuttosto che altre. Infatti misure come la betweenness non hanno restituito risultati significativi.
Sono stati comunque effettuati test utilizzando la rappresentazione a grafo per confrontare al meglio i risultati globali ottenuti.
 
\paragraph{\texttt{cs.illinois.edu}} Il grafo del sito presenta $728$ nodi e $16993$ archi. L'algoritmo \textit{Fastgreedy} non richiede parametri specifici ma è stato tagliato il dendrogramma ad altezza $15$, il numero di cluster nella operazione di raggruppamento manuale. Ugualmente per \textit{WalkTrap}, che però necessitava della lunghezza dei Random Walk da effettuare. I risultati migliori sono stati osservati con percorsi di lunghezza $3$.

 
\paragraph{\texttt{cs.stanford.edu}} Il grafo del sito presenta $1458$ nodi e $99686$ archi. Il dendrogramma restituito dall'algoritmo \textit{Fastgreedy} è stato troncato ad altezza $10$. Anche qui la scelta è ricaduta su Random Wlak di lunghezza $3$ per l'algoritmo \textit{WalkTrap}.
 
\paragraph{\texttt{eecs.mit.edu}} Il grafo del sito presenta $1745$ nodi e $63937$ archi. L'algoritmo \textit{Fastgreedy} ha costruito un dendrogramma che è stato tagliato ad altezza $15$, il numero di cluster nella operazione di raggruppamento manuale. Ugualmente per \textit{WalkTrap}, che però necessitava della lunghezza dei Random Walk da effettuare. I risultati migliori sono stati osservati con percorsi di lunghezza $3$.

\subsection{URL Embedding}
Considerando i Random Walk generati sul grafo come frasi, è possibile applicare algoritmi di Word Embedding per raggruppare le pagine sulla base del contesto in cui appaiono, ovvero le pagine che più verosimilmente appariranno insieme nelle sequenze.
\\
Le squenze di Random Walk sono state usate per apprendere rappresentazioni vettoriali delle pagine Web. La fase di URL embedding è stata effetuata utilizzando l'algoritmo Word2vec, \cite{gensim} (esaminato in \ref{word2vec}) modificando alcuni parametri e lasciando invariato altri.
\\
\begin{figure}[h!]
	\centering
	\includegraphics[width = 130mm]{lc_embedding_km.png}
	\caption{Rappresentazione del sito \texttt{cs.illinois.edu}, clusterizzato con K-Means.}
	\label{nc_embedding_km}
\end{figure}
I parametri personalizzati sono:
\begin{itemize}
\item \textbf{min-count}: tutte le parole (o URL) con frequenza di occorrenza minore i questo valore vengono ignorate.
\item \textbf{window} rappresenta la distanza massima tra l'URL corrente e quello predetto all'interno di una frase.
\item \textbf{negative} Nella fase di embedding di una URL, viene calcolato il rapporto tra la similarità del contesto con la parola e la sommatoria di tutte le similarità tra la parola e gli altri contesti. Più precisamente:
\begin{equation}
\frac{v_c \cdot v_w}{\sum\limits_{c \in C} v_c \cdot v_w}
\end{equation}
Questa operazione può essere molto lenta. Per accelerare il processo possono venir scelti $n$ contesti casuali da confrontare. Questo parametro, se maggiore di $0$, rappresenta il numero di vettori da confrontare.
\item \textbf{sg} definisce l'algoritmo di apprendimento, di default viene usato \textit{CBOW}, mentre se impostato a $1$ utilizza \textit{skip-gram} \cite{Mikolov13}
\end{itemize}
I migliori risultati sono stati osservati con:

\paragraph{\texttt{cs.illinois.edu}} è stata impostato un \textit{min-count} pari a $1$, \textit{window} con valore $5$ ed è stato utilizzato \textit{skip-gram} con $5$ \textit{negative} sampling. Gli algoritmi testati sono stati: DBSCAN con $\epsilon = 0.9$ ed \textit{min-samples}$ = 4$; HDBSCAN con \textit{min-cluster-size}$=6$; K-Means con numero di cluster pari a $15$. 

\paragraph{\texttt{cs.stanford.edu}} è stata impostato un \textit{min-count} pari a $1$, \textit{window} con valore $5$ ed è stato utilizzato \textit{skip-gram} con $5$ \textit{negative} sampling. Gli algoritmi testati sono stati: DBSCAN con $\epsilon = 0.9$ ed \textit{min-samples}$ = 7$; HDBSCAN con \textit{min-cluster-size}$=10$; K-Means con numero di cluster pari a $15$. Per il dataset delle liste: DBSCAN con $\epsilon = 1.6$ ed \textit{min-samples}$ = 3$; HDBSCAN con \textit{min-cluster-size}$=5$; K-Means con numero di cluster pari a $15$.

\paragraph{\texttt{eecs.mit.edu}} è stata impostato un \textit{min-count} pari a $1$, \textit{window} con valore $5$ ed è stato utilizzato \textit{skip-gram} con $5$ \textit{negative} sampling. Gli algoritmi testati sono stati: DBSCAN con $\epsilon = 1.0$ ed \textit{min-samples}$ = 10$; HDBSCAN con \textit{min-cluster-size}$=10$; K-Means con numero di cluster pari a $10$. Per il dataset delle liste: DBSCAN con $\epsilon = 1.5$ ed \textit{min-samples}$ = 5$; HDBSCAN con \textit{min-cluster-size}$=13$; K-Means con numero di cluster pari a $10$.


\subsection{Text Mining}
Sono state utilizzate tecniche di Text Mining per il clustering basato sul contenuto testuale. I contenuti all'interno di uno stesso sito web avranno una struttura e termini comuni, differenziandosi al variare dell'argomento trattato. La struttura gerarchica di un sito web organizza solitamente le pagine in sezioni simili. Questa metodologia tuttavia, considera solo l'informazione testuale, assumendo che i termini all'interno del sito web siano indipendenti l'uno dall'altro così come i documenti, ignorando le relazioni interdipendenti tra questi. Il web si discosta dall'analisi classica dei documenti proprio per le relazioni che intercorrono tra le pagine, tuttavia l'analisi testuale rimane molto importante.
\\
Nella fase di sperimentazione è stata utilizzata una rappresentazione vettoriale della frequenza dei termini all'interno dell'insieme delle pagine web, calcolata con la tecnica della \textit{frequency–inverse document frequency} (tf-idf).

I parametri personalizzati per la costruzione dell matrice documenti-termini con funzione di peso \textit{idf} sono stati:
\begin{itemize}
\item \textbf{max-df}: questo valore rappresenta la massima frequenza, all'interno dei documenti, che un termine può avere per essere utilizzato nella matrice tf-idf. Se un termine appare molte volte nel corpus, molto probabilmente avrà poco significato.
\item \textbf{min-df}: indica il numero minimo di documenti in cui un termine dovrà apparire per essere considerato.
\item \textbf{ngram-range}: vengono presi in considerazioni gli n-grammi di lunghezza compresa nell'intervallo specificato in questo parametro. Un n-gramma è una sottosequenza  di $n$ elementi di un'altra.
\end{itemize}

I risultati mostrati sono stati ottenuti nel seguente modo:
\paragraph{\texttt{cs.illinois.edu}}
Sul dataset del sito, costituito da $728$ pagine e $433$ termini, il corpus è stato ripulito delle stopword, stemmatizzato ed è stato impostato il \textit{max-df} all'80\%, il \textit{min-df} a $0.1$ e sono stati considerati solo uni-grammi, bi-grammi e tri-grammi. Se i termini appaiono in più dell'80\% dei documenti, probabilmente avrà poco significato, lo stesso se appare troppe poche volte. Gli algoritmi testati sono stati: DBSCAN con $\epsilon = 0.9$ ed \textit{min-samples}$ = 4$; HDBSCAN con \textit{min-cluster-size}$=4$; K-Means con numero di cluster pari a $15$. \\Sul Dataset costruito estraendo le liste: DBSCAN con $\epsilon = 0.7$ ed \textit{min-samples}$ = 4$; HDBSCAN con \textit{min-cluster-size}$=7$; K-Means con numero di cluster pari a $15$. 

\paragraph{\texttt{cs.stanford.edu}}
Sul dataset del sito, costituito da $1458$ pagine e $843$ termini, il corpus è stato ripulito delle stopword, stemmatizzato ed è stato impostato il \textit{max-df} all'80\%, il \textit{min-df} a $0.1$ e sono stati considerati solo uni-grammi, bi-grammi e tri-grammi. Se i termini appaiono in più dell'80\% dei documenti, probabilmente avrà poco significato, lo stesso se appare troppe poche volte. Gli algoritmi testati sono stati: DBSCAN con $\epsilon = 0.3$ ed \textit{min-samples}$ = 5$; HDBSCAN con \textit{min-cluster-size}$=15$; K-Means con numero di cluster pari a $15$. \\Sul Dataset costruito estraendo le liste: DBSCAN con $\epsilon = 0.5$ ed \textit{min-samples}$ = 3$; HDBSCAN con \textit{min-cluster-size}$=5$; K-Means con numero di cluster pari a $15$. 

\paragraph{\texttt{eecs.mit.edu}}
Sul dataset del sito, costituito da $1745$ pagine e $354$ termini, il corpus è stato ripulito delle stopword, stemmatizzato ed è stato impostato il \textit{max-df} all'80\%, il \textit{min-df} a $0.1$ e sono stati considerati solo uni-grammi, bi-grammi e tri-grammi. Se i termini appaiono in più dell'80\% dei documenti, probabilmente avrà poco significato, lo stesso se appare troppe poche volte. Gli algoritmi testati sono stati: DBSCAN con $\epsilon = 0.9$ ed \textit{min-samples}$ = 9$; HDBSCAN con \textit{min-cluster-size}$=15$; K-Means con numero di cluster pari a $10$. \\Sul Dataset costruito estraendo le liste: DBSCAN con $\epsilon = 0.9$ ed \textit{min-samples}$ = 6$; HDBSCAN con \textit{min-cluster-size}$=9$; K-Means con numero di cluster pari a $10$. 

\subsection{Embedding e Text Mining}
I risultati hanno evidenziato che l'analisi singola, sia della correlazione tra le pagine sia del contenuto testuale, può non bastare a codificare esaustivamente la conoscenza che una pagina web può offrire. Entrambe le informazioni sono rilevanti ed andrebbero processate combinatamente. Effettuando i test precedenti è stato osservato come le informazioni codificate nelle due tipologie di vettori fossero complementari. Sono stati quindi considerati come un unico vettore. Il vantaggio di associare le relazioni in uno spazio vettoriale offre il vantaggio usare la stessa rappresentazione e quindi di unire i vettori derivanti dagli algoritmi di word embedding con quelli derivanti dall'analisi di contenuto testuale. 
\\
Così facendo è possibile dare più importanza ad una tipologia di informazione piuttosto che ad un altra, andando a modificare il rapporto tra le dimensioni dei vettori.


\paragraph{\texttt{cs.illinois.edu}} I vettori di word2vec sono stati generati di dimensione $48$, mentre i vettori-riga documenti sono stati ridotti con \textit{TruncateSVD} a dimensione $50$. Gli algoritmi testati sono stati: HDBSCAN con \textit{min-cluster-size}$=7$; K-Means con numero di cluster pari a $15$. 

\paragraph{\texttt{cs.stanford.edu}} I vettori di word2vec sono stati generati di dimensione $48$, mentre i vettori-riga documenti sono stati ridotti con \textit{TruncateSVD} a dimensione $50$. Gli algoritmi testati sono stati: HDBSCAN con \textit{min-cluster-size}$=10$; K-Means con numero di cluster pari a $15$. 

\paragraph{\texttt{eecs.mit.edu}} I vettori di word2vec sono stati generati di dimensione $48$, mentre i vettori-riga documenti sono stati ridotti con \textit{TruncateSVD} a dimensione $50$. Gli algoritmi testati sono stati: HDBSCAN con \textit{min-cluster-size}$=14$; K-Means con numero di cluster pari a $10$. 

\section{Metriche}
Valutare le performance di un algoritmo di clustering non è banale come contare il numero di errori o calcolare metriche quali la precision e la recall di un algoritmo di apprendimento supervisionato. In particolare, le metriche di valutazione non dovrebbero prendere in considerazione gli specifici valori delle label. Piuttosto dovrebbero considerare se il raggruppamento generato dall'algortimo definisce una separazione dei dati similmente a quanto fornito nella \textit{ground truth}, ovvero il vero valore delle label, o soddisfare qualche assunzione, come ad esempio che membri dello stesso cluster siano più simili rispetto a quelli di cluster differenti, utilizzando una data funzione di similarità.

\begin{itemize}
\item \textbf{Homogeneity}
Nota la ground truth, questo valore rappresenta quanto ogni cluster restituito dall'apprendimento sia omogeneo, ovvero che contiene solo membri di una classe. 
\begin{equation}
h = 1 - \frac{H(C|K)}{H(C)}
\end{equation}
Dove $H(C|K)$ è l'entropia condizionale delle classi date le assegnazioni dei cluster:
\begin{equation}
H(C|K) = - \sum\limits_{c=1}^{|C|} \sum\limits_{k=1}^{|K|} \frac{n_{c,k}{n}} \cdot \log \left( \frac{n_{c,k}}{n_k}\right)
\end{equation}
\label{hck}
e $H(C)$ è l'entropia delle classi:
\begin{equation}
H(C) = \sum\limits_{c=1}^{|C|} \frac{n_c}{n} \cdot \log \left( \frac{n_{c}}{n}\right)
\end{equation}
\label{hc}
con $n$ il numero totale delle pagine, e $n_c$ ed $n_k$ il numero delle pagine appartenenti alla classe $c$ o al cluster $k$, ed infine $n_c,k$ le pagine della classe $c$ assegnate al cluster $k$.
\item \textbf{Completeness}
Note la ground truth, indica se tutti i membri di una sono stati assegnati allo stesso cluster. Da notare che se ad esempio tutte le pagine fossero asegnate ad un unico grande cluster, la \textit{Completeness} sarebbe massima.
\begin{equation}
c = 1 - \frac{H(C|K)}{H(K)}
\end{equation}
Dove $H(C|K)$ è la \ref{hck} e $H(K)$ è l'entropia dei cluster.
\item \textbf{V-Measure} rappresenta la media armonica fra l'\textit{homogeneity score} e il \textit{completeness score}.
\begin{equation}
v = 2 \cdot \frac{h \cdot c}{h + c}
\end{equation}

\item \textbf{Adjusted rand index}
Nota la ground truth, ovvero le classi reali, e le assegnazioni di un algritmo di apprendimento, viene calcolata una funzione che misura la similarità delle due informazioni, ignorandole permutazioni. I valori che può assumere vanno da $-1$ a $1$. Vicino allo $0$ rappresentano una assegnazione casuale delle etichette.
\\
Sia $C$ è gli assegnamenti della ground truth e $K$ le label predette, allora:
\\
$a$ è il numero di coppie di elementi che si trovano sia in $c$ che in $K$
\\
$b$ è il numero di coppie di elementi che si trovano in insiemi diversi in $C$ e in insiemi diversi in $K$.
Il \textbf{Random Index} è dato da:
\begin{equation}
RI = \frac{a + b}{C_2^n}
\end{equation}
Dove $C_2^n$ è il numero totale di tutte le possibili coppie nel dataset. 
\\
Il $RI$ non garantisce comunque che le assegnazioni casuali avranno valori prossimi allo $0$.
\\ Viene definito L'Adjusted Random Index:
\begin{equation}
ARI = \frac{RI - E[RI]}{\max (RI) - E[RI]}
\end{equation}

\item \textbf{Mutual Information}
Nota la ground truth e le asseganzioni dell'algoritmo di clustering, la \textit{Mutual Information} è una funzione che misura la corrispondenza delle due informazioni, ignorando le permutazioni. 
\\
Anche qui, una assegnazione uniforme (o casuale) dei cluster avrà valori prossimi allo $0.0$, evidenziando l'indipendenza delle due informazioni, mentre valori prossimi a $1.0$ indicano una corrispondenza significativa.
\\
Dati due assegnamenti di pari lunghezza, $U$ e $V$, la loro entropia è la quantità di incertezza per una partizione , definita da:
\begin{equation}
H(U) = \sum\limits_{i=1}^{|U|} P(i)\log (P(i))
\end{equation}
Dove $P(i) = \frac{|U_i|}{N}$ è la probabilità che un oggetto preso a caso da $U$ appartenga alla classe $U_i$. Similmente per $V$:
\begin{equation}
H(V) = \sum\limits_{j=1}^{|V|} P'(j)\log (P'(j))
\end{equation}
Con $P'(j) = \frac{|V_j|}{N}$. La Mutual Information p definita come:
\begin{equation}
MI(U,V) = \sum\limits_{i=1}^{|U|}\sum\limits_{j=1}^{|V|} P(i,j)\log \left( \frac{P(i,j)}{P(i)P'(j)} \right)
\end{equation}

\end{itemize}


Nelle metriche presentate, fatta eccezione per l'Adjusted Rand Index, $0.0$ rappresenta il valore peggiore, mentre $1.0$ il perfect score. Questi valori offrono una interpretazione intuitiva e può aiutare alla scoperta degli errori commessi nella assegnazione. Fra i vantaggi è degno di nota che nessuna assunzione viene fatta sulla struttura dei cluster, quindi possono essere utilizzate con algoritmi che identificano cluster di forma diversa.

\paragraph{Silhouette}
Se la Ground Truth non è nota, può essere usata la Silhouette usando il modello stesso. Ad un alto valore corrispondono modelli con cluster ben definiti. Si compone di due valore
\begin{itemize}
\item $a$ : la distanza media tra un vettore e tutti gli altri nella stessa classe
\item $b$ : la distanza media tra un vettore e tutti gli altri nella classe più vicina
\end{itemize}

Il coefficiente di \textit{Silhouette} è dato quindi da:
\begin{equation}
s = \frac{b - a}{\max (a,b)}
\end{equation}

Il valore può variare da $-1$, per cluster non definiti, a $1$ per cluster densamente connessi. Intorno allo $0$ indica cluster sovrapposti. Tende comunque ad essere maggiore per cluster convessi, o con alta densità, come quelli ottenuti con DBSCAN.

\section{Risultati}

\subsubsection{Community Detection}

I grafi utilizzati rappresentano le due diverse operazioni di estrazione di collegamenti effettuate. La prima contiene tutti i collegamenti presenti all'interno di una pagina e mostrerà quindi più archi. La seconda estrae solamente gli hyperlink dalle liste.
\color{red} devo inserire dei commenti prima e dopo ogni tab \color{black}
\begin{table}[H]
	\begin{tabular}{| l | c | c | c | c | c |}
	\hline
	\textbf{Graph}  & \textbf{Homog} & \textbf{Compl} & \textbf{V-Measure}  & \textbf{ARI}  & \textbf{MI} \\ [3ex] \hline
	\textbf{nc WalkTrap} & 0.6471 & 0.6585 & 0.6527 & 0.4363 & 0.6281\\ [3ex]
	 \hline
	\textbf{nc Fastgreedy} & 0.5518 & 0.8563 & 0.6711 & 0.5764 & 0.5354\\ [3ex]
	 \hline	
	\textbf{lc WalkTrap} & 0.5093 & 0.4892 & 0.4991 & 0.2762 & 0.4722\\ [3ex]
	 \hline	
	\textbf{lc Fastgreedy} & 0.5522 & 0.6035 & 0.5767 & 0.3656 & 0.5382\\ [3ex]
	\hline
	\end{tabular}
	\caption{Risultati sperimentazione di partizionamento del grafo del sito \texttt{cs.illinois.edu}}
	\label{metricheGraphIll}
\end{table}

\begin{table}[H]
	\begin{tabular}{| l | c | c | c | c | c |}
	\hline
	\textbf{Graph}  & \textbf{Homog} & \textbf{Compl} & \textbf{V-Measure}  & \textbf{ARI}  & \textbf{MI} \\ [3ex] \hline
	\textbf{nc WalkTrap} & 0.6471 & 0.6585 & 0.6527 & 0.4363 & 0.6281\\ [3ex]
	 \hline
	\textbf{nc Fastgreedy} & 0.5518 & 0.8563 & 0.6711 & 0.5764 & 0.5354\\ [3ex]
	 \hline	
	\textbf{lc WalkTrap} & 0.5093 & 0.4892 & 0.4991 & 0.2762 & 0.4722\\ [3ex]
	 \hline	
	\textbf{lc Fastgreedy} & 0.5522 & 0.6035 & 0.5767 & 0.3656 & 0.5382\\ [3ex]
	\hline
	\end{tabular}
	\caption{Risultati sperimentazione di partizionamento del grafo del sito \texttt{cs.illinois.edu}}
	\label{metricheGraphIll}
\end{table}

\begin{table}[H]
	\begin{tabular}{| l | c | c | c | c | c |}
	\hline
	\textbf{Graph}  & \textbf{Homog} & \textbf{Compl} & \textbf{V-Measure}  & \textbf{ARI}  & \textbf{MI} \\ [3ex] \hline
	\textbf{nc WalkTrap} & 0.6471 & 0.6585 & 0.6527 & 0.4363 & 0.6281\\ [3ex]
	 \hline
	\textbf{nc Fastgreedy} & 0.5518 & 0.8563 & 0.6711 & 0.5764 & 0.5354\\ [3ex]
	 \hline	
	\textbf{lc WalkTrap} & 0.5093 & 0.4892 & 0.4991 & 0.2762 & 0.4722\\ [3ex]
	 \hline	
	\textbf{lc Fastgreedy} & 0.5522 & 0.6035 & 0.5767 & 0.3656 & 0.5382\\ [3ex]
	\hline
	\end{tabular}
	\caption{Risultati sperimentazione di partizionamento del grafo del sito \texttt{cs.illinois.edu}}
	\label{metricheGraphIll}
\end{table}

L'analisi del grafo considera unicamente le relazioni che intercorrono fra le pagine web e tralascia informazioni riguardanti il contenuto. Dalle metriche rilevate risulta in tabella \ref{metricheGraphIll}, risulta che il partizionamento del grafo web non riesce a dividere al meglio i cluster.


\subsubsection{URL Embedding}

\begin{table}[H]
	\begin{tabular}{| l | c | c | c | c | c | c |}
	\hline
	\textbf{Embed}  & \textbf{Homog} & \textbf{Compl} & \textbf{V-Meas}  & \textbf{ARI}  & \textbf{MI}  & \textbf{Silh} \\ [3ex] \hline
	\textbf{nc dbscan} & 0.5553 & 0.6579 & 0.6023 & 0.4487 & 0.5234 & 0.2588\\ [3ex]
	 \hline 
	\textbf{nc hdbscan} & 0.5759 & 0.6720 & 0.6203 & 0.5282 & 0.5525 & 0.2573\\ [3ex]
	 \hline
	\textbf{nc Kmeans} & 0.8238 & 0.7575 & 0.7892 & 0.7883 & 0.7423 & 0.3131\\ [3ex]
	 \hline	
	\textbf{lc dbscan} & 0.4163 & 0.5922 & 0.4889 & 0.2250 & 0.3935 & 0.1320\\ [3ex]
	\hline
	\textbf{lc hdbscan} & 0.4760 & 0.5067 & 0.4908 & 0.2275 & 0.4515 & 0.1054\\ [3ex]
	\hline
	
	\textbf{lc Kmeans} & 0.8095 & 0.6593 & 0.7267 & 0.6189 & 0.6473 & 0.2281\\ [3ex]
	\hline
	\end{tabular}
	\caption{Risultati sperimentazione di partizionamento del grafo del sito \texttt{cs.illinois.edu}}
	\label{metricheEmbed}
\end{table}

\begin{table}[H]
	\begin{tabular}{| l | c | c | c | c | c | c |}
	\hline
	\textbf{Embed}  & \textbf{Homog} & \textbf{Compl} & \textbf{V-Meas}  & \textbf{ARI}  & \textbf{MI}  & \textbf{Silh} \\ [3ex] \hline
	\textbf{nc dbscan} & 0.5553 & 0.6579 & 0.6023 & 0.4487 & 0.5234 & 0.2588\\ [3ex]
	 \hline 
	\textbf{nc hdbscan} & 0.5759 & 0.6720 & 0.6203 & 0.5282 & 0.5525 & 0.2573\\ [3ex]
	 \hline
	\textbf{nc Kmeans} & 0.8238 & 0.7575 & 0.7892 & 0.7883 & 0.7423 & 0.3131\\ [3ex]
	 \hline	
	\textbf{lc dbscan} & 0.4163 & 0.5922 & 0.4889 & 0.2250 & 0.3935 & 0.1320\\ [3ex]
	\hline
	\textbf{lc hdbscan} & 0.4760 & 0.5067 & 0.4908 & 0.2275 & 0.4515 & 0.1054\\ [3ex]
	\hline
	
	\textbf{lc Kmeans} & 0.8095 & 0.6593 & 0.7267 & 0.6189 & 0.6473 & 0.2281\\ [3ex]
	\hline
	\end{tabular}
	\caption{Risultati sperimentazione di partizionamento del grafo del sito \texttt{cs.illinois.edu}}
	\label{metricheEmbed}
\end{table}

\begin{table}[H]
	\begin{tabular}{| l | c | c | c | c | c | c |}
	\hline
	\textbf{Embed}  & \textbf{Homog} & \textbf{Compl} & \textbf{V-Meas}  & \textbf{ARI}  & \textbf{MI}  & \textbf{Silh} \\ [3ex] \hline
	\textbf{nc dbscan} & 0.5553 & 0.6579 & 0.6023 & 0.4487 & 0.5234 & 0.2588\\ [3ex]
	 \hline 
	\textbf{nc hdbscan} & 0.5759 & 0.6720 & 0.6203 & 0.5282 & 0.5525 & 0.2573\\ [3ex]
	 \hline
	\textbf{nc Kmeans} & 0.8238 & 0.7575 & 0.7892 & 0.7883 & 0.7423 & 0.3131\\ [3ex]
	 \hline	
	\textbf{lc dbscan} & 0.4163 & 0.5922 & 0.4889 & 0.2250 & 0.3935 & 0.1320\\ [3ex]
	\hline
	\textbf{lc hdbscan} & 0.4760 & 0.5067 & 0.4908 & 0.2275 & 0.4515 & 0.1054\\ [3ex]
	\hline
	
	\textbf{lc Kmeans} & 0.8095 & 0.6593 & 0.7267 & 0.6189 & 0.6473 & 0.2281\\ [3ex]
	\hline
	\end{tabular}
	\caption{Risultati sperimentazione di partizionamento del grafo del sito \texttt{cs.illinois.edu}}
	\label{metricheEmbed}
\end{table}

\subsubsection{Text Mining}

\begin{table}[H]
	\begin{tabular}{| l | c | c | c | c | c | c |}
	\hline
	\textbf{Embed}  & \textbf{Homog} & \textbf{Compl} & \textbf{V-Meas}  & \textbf{ARI}  & \textbf{MI}  & \textbf{Silh} \\ [3ex] \hline
	\textbf{nc dbscan} & 0.5553 & 0.6579 & 0.6023 & 0.4487 & 0.5234 & 0.2588\\ [3ex]
	 \hline 
	\textbf{nc hdbscan} & 0.5759 & 0.6720 & 0.6203 & 0.5282 & 0.5525 & 0.2573\\ [3ex]
	 \hline
	\textbf{nc Kmeans} & 0.8238 & 0.7575 & 0.7892 & 0.7883 & 0.7423 & 0.3131\\ [3ex]
	 \hline	
	\textbf{lc dbscan} & 0.4163 & 0.5922 & 0.4889 & 0.2250 & 0.3935 & 0.1320\\ [3ex]
	\hline
	\textbf{lc hdbscan} & 0.4760 & 0.5067 & 0.4908 & 0.2275 & 0.4515 & 0.1054\\ [3ex]
	\hline
	
	\textbf{lc Kmeans} & 0.8095 & 0.6593 & 0.7267 & 0.6189 & 0.6473 & 0.2281\\ [3ex]
	\hline
	\end{tabular}
	\caption{Risultati sperimentazione di partizionamento del grafo del sito \texttt{cs.illinois.edu}}
	\label{metricheEmbed}
\end{table}


\begin{table}[H]
	\begin{tabular}{| l | c | c | c | c | c | c |}
	\hline
	\textbf{Embed}  & \textbf{Homog} & \textbf{Compl} & \textbf{V-Meas}  & \textbf{ARI}  & \textbf{MI}  & \textbf{Silh} \\ [3ex] \hline
	\textbf{nc dbscan} & 0.5553 & 0.6579 & 0.6023 & 0.4487 & 0.5234 & 0.2588\\ [3ex]
	 \hline 
	\textbf{nc hdbscan} & 0.5759 & 0.6720 & 0.6203 & 0.5282 & 0.5525 & 0.2573\\ [3ex]
	 \hline
	\textbf{nc Kmeans} & 0.8238 & 0.7575 & 0.7892 & 0.7883 & 0.7423 & 0.3131\\ [3ex]
	 \hline	
	\textbf{lc dbscan} & 0.4163 & 0.5922 & 0.4889 & 0.2250 & 0.3935 & 0.1320\\ [3ex]
	\hline
	\textbf{lc hdbscan} & 0.4760 & 0.5067 & 0.4908 & 0.2275 & 0.4515 & 0.1054\\ [3ex]
	\hline
	
	\textbf{lc Kmeans} & 0.8095 & 0.6593 & 0.7267 & 0.6189 & 0.6473 & 0.2281\\ [3ex]
	\hline
	\end{tabular}
	\caption{Risultati sperimentazione di partizionamento del grafo del sito \texttt{cs.illinois.edu}}
	\label{metricheEmbed}
\end{table}


\begin{table}[H]
	\begin{tabular}{| l | c | c | c | c | c | c |}
	\hline
	\textbf{Embed}  & \textbf{Homog} & \textbf{Compl} & \textbf{V-Meas}  & \textbf{ARI}  & \textbf{MI}  & \textbf{Silh} \\ [3ex] \hline
	\textbf{nc dbscan} & 0.5553 & 0.6579 & 0.6023 & 0.4487 & 0.5234 & 0.2588\\ [3ex]
	 \hline 
	\textbf{nc hdbscan} & 0.5759 & 0.6720 & 0.6203 & 0.5282 & 0.5525 & 0.2573\\ [3ex]
	 \hline
	\textbf{nc Kmeans} & 0.8238 & 0.7575 & 0.7892 & 0.7883 & 0.7423 & 0.3131\\ [3ex]
	 \hline	
	\textbf{lc dbscan} & 0.4163 & 0.5922 & 0.4889 & 0.2250 & 0.3935 & 0.1320\\ [3ex]
	\hline
	\textbf{lc hdbscan} & 0.4760 & 0.5067 & 0.4908 & 0.2275 & 0.4515 & 0.1054\\ [3ex]
	\hline
	
	\textbf{lc Kmeans} & 0.8095 & 0.6593 & 0.7267 & 0.6189 & 0.6473 & 0.2281\\ [3ex]
	\hline
	\end{tabular}
	\caption{Risultati sperimentazione di partizionamento del grafo del sito \texttt{cs.illinois.edu}}
	\label{metricheEmbed}
\end{table}

\subsubsection{Embedding e Text Mining}

\begin{table}[H]
	\begin{tabular}{| l | c | c | c | c | c | c |}
	\hline
	\textbf{Embed}  & \textbf{Homog} & \textbf{Compl} & \textbf{V-Meas}  & \textbf{ARI}  & \textbf{MI}  & \textbf{Silh} \\ [3ex] \hline
	\textbf{nc dbscan} & 0.5553 & 0.6579 & 0.6023 & 0.4487 & 0.5234 & 0.2588\\ [3ex]
	 \hline 
	\textbf{nc hdbscan} & 0.5759 & 0.6720 & 0.6203 & 0.5282 & 0.5525 & 0.2573\\ [3ex]
	 \hline
	\textbf{nc Kmeans} & 0.8238 & 0.7575 & 0.7892 & 0.7883 & 0.7423 & 0.3131\\ [3ex]
	 \hline	
	\textbf{lc dbscan} & 0.4163 & 0.5922 & 0.4889 & 0.2250 & 0.3935 & 0.1320\\ [3ex]
	\hline
	\textbf{lc hdbscan} & 0.4760 & 0.5067 & 0.4908 & 0.2275 & 0.4515 & 0.1054\\ [3ex]
	\hline
	
	\textbf{lc Kmeans} & 0.8095 & 0.6593 & 0.7267 & 0.6189 & 0.6473 & 0.2281\\ [3ex]
	\hline
	\end{tabular}
	\caption{Risultati sperimentazione di partizionamento del grafo del sito \texttt{cs.illinois.edu}}
	\label{metricheEmbed}
\end{table}


\begin{table}[H]
	\begin{tabular}{| l | c | c | c | c | c | c |}
	\hline
	\textbf{Embed}  & \textbf{Homog} & \textbf{Compl} & \textbf{V-Meas}  & \textbf{ARI}  & \textbf{MI}  & \textbf{Silh} \\ [3ex] \hline
	\textbf{nc dbscan} & 0.5553 & 0.6579 & 0.6023 & 0.4487 & 0.5234 & 0.2588\\ [3ex]
	 \hline 
	\textbf{nc hdbscan} & 0.5759 & 0.6720 & 0.6203 & 0.5282 & 0.5525 & 0.2573\\ [3ex]
	 \hline
	\textbf{nc Kmeans} & 0.8238 & 0.7575 & 0.7892 & 0.7883 & 0.7423 & 0.3131\\ [3ex]
	 \hline	
	\textbf{lc dbscan} & 0.4163 & 0.5922 & 0.4889 & 0.2250 & 0.3935 & 0.1320\\ [3ex]
	\hline
	\textbf{lc hdbscan} & 0.4760 & 0.5067 & 0.4908 & 0.2275 & 0.4515 & 0.1054\\ [3ex]
	\hline
	
	\textbf{lc Kmeans} & 0.8095 & 0.6593 & 0.7267 & 0.6189 & 0.6473 & 0.2281\\ [3ex]
	\hline
	\end{tabular}
	\caption{Risultati sperimentazione di partizionamento del grafo del sito \texttt{cs.illinois.edu}}
	\label{metricheEmbed}
\end{table}


\begin{table}[H]
	\begin{tabular}{| l | c | c | c | c | c | c |}
	\hline
	\textbf{Embed}  & \textbf{Homog} & \textbf{Compl} & \textbf{V-Meas}  & \textbf{ARI}  & \textbf{MI}  & \textbf{Silh} \\ [3ex] \hline
	\textbf{nc dbscan} & 0.5553 & 0.6579 & 0.6023 & 0.4487 & 0.5234 & 0.2588\\ [3ex]
	 \hline 
	\textbf{nc hdbscan} & 0.5759 & 0.6720 & 0.6203 & 0.5282 & 0.5525 & 0.2573\\ [3ex]
	 \hline
	\textbf{nc Kmeans} & 0.8238 & 0.7575 & 0.7892 & 0.7883 & 0.7423 & 0.3131\\ [3ex]
	 \hline	
	\textbf{lc dbscan} & 0.4163 & 0.5922 & 0.4889 & 0.2250 & 0.3935 & 0.1320\\ [3ex]
	\hline
	\textbf{lc hdbscan} & 0.4760 & 0.5067 & 0.4908 & 0.2275 & 0.4515 & 0.1054\\ [3ex]
	\hline
	
	\textbf{lc Kmeans} & 0.8095 & 0.6593 & 0.7267 & 0.6189 & 0.6473 & 0.2281\\ [3ex]
	\hline
	\end{tabular}
	\caption{Risultati sperimentazione di partizionamento del grafo del sito \texttt{cs.illinois.edu}}
	\label{metricheEmbed}
\end{table}

\subsection{Analisi dei risultati}
\textbf{K-Means - cs.illinois.edu}
\begin{table}[H]
	\begin{tabular}{| l | c | c | c | c | c | c |}
	\hline
	\textbf{Type}  & \textbf{Homog} & \textbf{Compl} & \textbf{V-Meas}  & \textbf{ARI}  & \textbf{MI}  & \textbf{Silh} \\ [3ex] \hline
	\textbf{Embedding} & 0.5553 & 0.6579 & 0.6023 & 0.4487 & 0.5234 & 0.2588\\ [3ex]
	 \hline 
	\textbf{Text Mining} & 0.5759 & 0.6720 & 0.6203 & 0.5282 & 0.5525 & 0.2573\\ [3ex]
	 \hline
	\textbf{Emb + Text} & 0.8238 & 0.7575 & 0.7892 & 0.7883 & 0.7423 & 0.3131\\ [3ex]
	 \hline
	\end{tabular}
	\caption{Risultati sperimentazione di partizionamento del grafo del sito \texttt{cs.illinois.edu}}
	\label{metricheEmbed}
\end{table}

Lo scopo della tesi non era valutare l'efficacia dei vari algoritmi riportati, ma verificare un eventuale miglioramento nei risultati ottenuti.
\\
Dalle metriche rilevate è emerso che l'uso delle liste non ha influenzato particolarmente i risultati ottenuti. Sono stati riscontrati miglioramenti dell'$n$\% per quanto riguarda 'Homogeneity. ecc \color{red} (poi lo scrivo e ci metto le altre tabelle raggruppate per algoritmo) \color{black}
\\\\
Valutare i risultati di un algoritmo di clustering non è un operazione semplice. Infatti l'assegnazione manuale delle etichette denota una certa arbitrarietà. 
\\Inoltre l'analisi dei percorsi ha fatto notare come certe classi, idealmente raggruppate insieme in quanto stessa entità (e.g. docenti), possano invece essere divise in fase di apprendimento per motivi ragionevoli. Ad esempio, nel sito \textit{cs.illinois.edu}, erano presenti molteplici pagine relative agli stessi professori. Questo era dovuto al fatto che durante gli anni erano state pubblicate diverse edizioni del sito. Questo ha portato le diverse versioni della stessa pagina(concettuale) ad essere presenti in percorsi diversi. O ancora ad avere anche testo differente. Infatti anche l'analisi testuale fra le diverse versioni era differente. 
\\\\
Un altro esempio era il raggruppamento di pagine con poco testo. O ancora pagine relative agli studenti ''undergraduates'' ma in contesti differenti sono state etichettate allo stesso modo, dovuto al fatto che apparivano in percorsi simili ed avevano probabilmente testi simili.
\\\\
In conclusione il problema del clustering di pagine web può rivelarsi ostico e dare risultati diversi da quelli desiderati ma comunque sensati. Considerare più aspetti può essere rivelarsi utile in molti contesti applicativi.



\chapter{Conclusioni e sviluppi futuri}
% !TEX encoding = UTF-8
% !TEX TS-program = pdflatex
% !TEX root = ../Tesi.tex
% !TEX spellcheck = it-IT

%*******************************************************
% Introduzione
%*******************************************************
In questo lavoro di tesi è stato trattato il clustering di pagine Web, proponendo l'utilizzo dei Random Walk per apprendere rappresentazioni vettoriali delle pagine, utilizzato unitamente al loro contenuto testuale. Il lavoro non pretende di essere esaustivo, ma piuttosto un punto di partenza per ulteriori sviluppi, sia teorici che sperimentali. 

I risultati sperimentali prodotti si sono rivelati discreti e incentivano a proseguire gli studi in questa direzione in modo da individuare nuove tecniche che permettano di migliorare i risultati raggiunti in termini di qualità. 
In particolare è stato osservato come la forma dei cluster e le informazioni celate nei vari aspetti considerati; varino in funzione dal Dataset. Quindi, sarebbe opportuno utilizzare l'algoritmo più appropriato in base al contesto. 
\\
Da notare come le ''analogie'' estraibili dall'utilizzo di \textit{Word2vec} su collezioni di documenti, abbiano avuto un riscontro nel Web.

Lo scopo della tesi non era valutare l'efficacia dei vari algoritmi riportati, ma verificare un eventuale miglioramento nei risultati ottenuti attraverso l'applicazione del metodo proposto.
\\
Valutare i risultati di un algoritmo di clustering non è un operazione semplice. Infatti l'assegnazione manuale delle etichette denota una certa arbitrarietà. Sviluppi futuri potrebbero concentrarsi su di una analisi approfondita e molto più ampia delle performance dell'algoritmo.
Infatti, l'analisi dei percorsi ha fatto notare come certe classi, idealmente raggruppate insieme in quanto stessa entità (e.g. docenti), possano invece essere divise in fase di apprendimento per motivi ragionevoli.
\\\\
In conclusione il problema del clustering di pagine web può rivelarsi ostico e dare risultati diversi da quelli desiderati ma comunque sensati. Considerare più aspetti può essere rivelarsi utile in molti contesti applicativi.


%\input{Capitoli/Ipsum}
%\appendix
%\input{Capitoli/Dolor}
% *****************************************************************
% Materiale finale
%******************************************************************
%% !TEX encoding = UTF-8
% !TEX TS-program = pdflatex
% !TEX root = ../Tesi.tex
% !TEX spellcheck = it-IT

%*******************************************************
% Introduzione
%*******************************************************
In questo lavoro di tesi è stato trattato il clustering di pagine Web, proponendo l'utilizzo dei Random Walk per apprendere rappresentazioni vettoriali delle pagine, utilizzato unitamente al loro contenuto testuale. Il lavoro non pretende di essere esaustivo, ma piuttosto un punto di partenza per ulteriori sviluppi, sia teorici che sperimentali. 

I risultati sperimentali prodotti si sono rivelati discreti e incentivano a proseguire gli studi in questa direzione in modo da individuare nuove tecniche che permettano di migliorare i risultati raggiunti in termini di qualità. 
In particolare è stato osservato come la forma dei cluster e le informazioni celate nei vari aspetti considerati; varino in funzione dal Dataset. Quindi, sarebbe opportuno utilizzare l'algoritmo più appropriato in base al contesto. 
\\
Da notare come le ''analogie'' estraibili dall'utilizzo di \textit{Word2vec} su collezioni di documenti, abbiano avuto un riscontro nel Web.

Lo scopo della tesi non era valutare l'efficacia dei vari algoritmi riportati, ma verificare un eventuale miglioramento nei risultati ottenuti attraverso l'applicazione del metodo proposto.
\\
Valutare i risultati di un algoritmo di clustering non è un operazione semplice. Infatti l'assegnazione manuale delle etichette denota una certa arbitrarietà. Sviluppi futuri potrebbero concentrarsi su di una analisi approfondita e molto più ampia delle performance dell'algoritmo.
Infatti, l'analisi dei percorsi ha fatto notare come certe classi, idealmente raggruppate insieme in quanto stessa entità (e.g. docenti), possano invece essere divise in fase di apprendimento per motivi ragionevoli.
\\\\
In conclusione il problema del clustering di pagine web può rivelarsi ostico e dare risultati diversi da quelli desiderati ma comunque sensati. Considerare più aspetti può essere rivelarsi utile in molti contesti applicativi.

\bibliographystyle{plain}
\bibliography{Bibliografia}                % database di biblatex 

\end{document}
