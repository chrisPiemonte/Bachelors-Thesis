% !TEX encoding = UTF-8
% !TEX TS-program = pdflatex
% !TEX root = ../Tesi.tex
% !TEX spellcheck = it-IT

%*******************************************************
% Introduzione
%*******************************************************
In questo lavoro di tesi è stato trattato il clustering di pagine Web, proponendo una nuova metodologia.
La trattazione effettuata è incentrata sull'utilizzo dei Random Walk per apprendere rappresentazioni vettoriali delle pagine, unitamente al loro contenuto testuale. Il lavoro non pretende di essere esaustivo, ma piuttosto un punto di partenza per ulteriori sviluppi, sia teorici che sperimentali. 
\\\\
I risultati sperimentali prodotti si sono rivelati discreti e incentivano a proseguire gli studi in questa direzione in modo da individuare nuove tecniche che permettano di migliorare i risultati raggiunti in termini di qualità. 
In particolare è stato osservato come la forma dei cluster vari in funzione dal Dataset. Quindi, sarebbe opportuno utilizzare l'algoritmo più appropriato in base al contesto. 
\\
Da notare come le ''analogie'' estraibili dall'utilizzo di \textit{Word2vec} su collezioni di documenti, abbiano avuto un riscontro nel Web.